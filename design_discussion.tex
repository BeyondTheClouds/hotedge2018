
%%% Local Variables:
%%% mode: latex
%%% TeX-master: t
%%% End:

\section{Design Discussion}
\label{sec:design_discussion}

In this section, we discuss the pros and cons of the \emph{top-down} and
\emph{bottom-up} designs targeted at the end of the previous section to extend
OpenStack with multi-site collaboration. More precisely, our discussion is
driven by the following questions: (i) On the one hand, while the top-down
design is the most common approach, can it fulfill all the expected
requirements listed in our classification? (ii) On the other hand, while the
bottom-up design seems to disrupt the design principles of most existing VIMs,
should it be discarded?
 
\paragraph{Top-Down Design}
In the top-down design, the system relies on existing VIM APIs and overlay
components, with no modifications to the codebase of the VIM. This
absence of modifications is of particular importance in OpenStack,
since it provides a new release every six months with a lot of changes
in the codebase, whereas changes rarely impact the APIs. Therefore, a
top-down design gives independence to both VIM development and overlay
system components. If designed to do so, the system can easily allow
L4 requirements, \ie the support of different VIMs and different
versions of a VIM - \eg FogBow~\cite{brasileiro2016fogbow}.

However, requirements cannot be fulfilled without changing the core of
a VIM (\eg L2 type of requirements like live migration and virtual
networking across sites). For example, OpenStack
Tricircle~\cite{tricircle}, a project that aims to allow virtual
networking across different sites by introducing an overlay component,
ended up "breaking" the core of OpenStack by introducing non-standard
mechanisms to fulfill the requirements. Missing Edge relative VIM
features such as a fined authorization management with different
rights in different Edge sites means that many low-level VIM services
need to be reimplemented at the overlay level (\eg authentication,
resource scheduling, resource monitoring, etc). Ultimately, one might
end reimplementing repeatedly the entire low-level system at higher
levels.

%% Moreover, in order to operate a distributed Edge infrastructure through a single logical entry point, it means that many low-level VIM services need to be reimplemented at the overlay level (\eg authentication, resource scheduling, resource monitoring, etc). Ultimately, one might end reimplementing repeatedly the entire low level system at higher levels.



%An interesting fact about existing \emph{top-down} is that nearly all ...

%------Moreover, if different VIM flavors are deployed, it might become impractical to provide a single VIM management-view to users as different VIMs might have specific features that need to be managed.


\paragraph{Bottom-Up Design}
In bottom-up design, the system is not limited by what is
made available through VIM APIs. If one can make OpenStack natively
collaborative, this can allow supporting features across the entire
Edge infrastructure for free. For instance, the OpenStack ``boot a
VM'' process looks as follow from a bird's-eye view: (1) Get the URL
of the image by looking up into the database, (2) Schedule and boot
the VM. Thus, booting a VM on \emph{Site 1} using an image defined in
\emph{Site 2} is the same workflow as booting a VM on \emph{Site 1}
using an image defined in \emph{Site 1} if both sites shared the same
data (\ie database backend).

However, the efficiency and ease to support collaboration come at a
cost. OpenStack has not been designed with distribution, like one of
the edge scenarios, in mind. This kind of collaboration may be
achieved with a collaborative backend that relies on a notion of
\emph{space}. \emph{Space} says in which space the OpenStack lookup
and write data. It could be an Edge site or a combination of Edge
sites if the collaboration is needed. It leads to a more efficient
system, one without the need to repeatedly implement features at
different levels. But, implementing a backend based on a notion of
space represents a scientific challenge. With the high industry demand
for a working Edge system, this means that modifying or extending
OpenStack is in most cases too expensive (time and resource wise).
Note also that such backend will not free the developer to implement
specific workflows for intra-services scenarios\RACm{For instance the
  ``list of actually scheduling VM'' could not be reused from vanilla
  code}.

%Moreover, it can potentially fullfil any and all (reasonable) requirements since any feature missing can be implemented.
%% \JM{Ronan, can you add references of existing work?}

%% However, the efficiency and ease to support new features comes at a cost. Most existing systems, \eg OpenStack, were not designed with distribution, like the one of edge scenarios, in mind. With the high industry demand for a working Edge system, this means that modifying or extending existing systems (or even creating a new system) is in most cases too expensive (time and resource wise).
%% Moreover, a \emph{bottom-up} approach would certainly mean the entire Edge infrastructure would rely on a specific platform/VIM, which seems to be a unrealistic scenario today due to the still fragmented VIM landscape. On a short-term, a way to overcome this would be to have a complementary \emph{top-down} system. On a longer perspective, the ideal scenario would be for the industry to agree on a standardized system in a similar way as it has done for mobile communications with 3GPP~\cite{3gpp}.

\paragraph{Summary}
In the end, there are \emph{pros} and \emph{cons} in both approaches, and none individually seems to meet all technical and business requirements of edge scenarios.
On a technical perspective, a top-down approach brings modularity and independence between system components. However, it is limited by that same independence as it requires low-level features to be enabled to fulfill all requirements. Moreover, a full working system would be far from effient as several features need to be implemented at different levels.
A bottom-up approach can produce a more efficient system that can easily support additional requirements.
On a business perspective, a top-down approach can easily fulfil the short and medium-term Time-To-Market (TTM) requirements. In that sense, top-down is the working solution the community needs to aim for first, and is in fact the one it has been working towards so far. However, serious effort needs to be put on a long-term solution that lays on a bottom-up approach.

%% As a starting point, early \emph{bottom-up} efforts
%% \JM{Ronan, reference to your paper?}
%%  can be taken into account along with mechanisms that have been implemented in \emph{top-down} solutions but can be applied in a \emph{bottom-up} way~\cite{ericsson-p2p}.


%% \JM{Would a picture somewhere in the paper with a simplified view of a top-down system vs a bottom-up one help the reader? Maybe in section 3.3? I think it would be useful, but we would need to cut somewhere to make it fit in the 5 pages}

%NOTES:
%The way to go is bottom up. Top-down, fastest TTM, a quick fix. That is how everyone is doing it, and thatis the way. BUt that is no the way it should be designed. Short-medium term, top-down...long-run, bottom-up.
%Refer to:
%http://ieeexplore.ieee.org/abstract/document/1334777/
%Top-down: requirements, then implement it
%Bottom-up: not all requirments, but implement something. Can top-down approaches be partially used and implemented in a bottom-up.
