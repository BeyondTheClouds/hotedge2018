%%% Local Variables:
%%% mode: latex
%%% TeX-master: t
%%% End:

\section{Design Discussion}
\label{sec:design_discussion}

The top-down and bottom-up designs introduced in the previous section have 
their own pros and cons. 
% to extend OpenStack with multi-site collaboration.
%More precisely, our discussion is driven by the following questions: (i) On the
%one hand, while the top-down design is the most common approach, can it fulfill
%all the expected requirements listed in our classification without requiring
%changes in the VIM codebase? (ii) On the other hand, while the bottom-up design
%seems to disrupt the design principles by requiring \emph{a la peer-to-peer}
%strategies in VIM's internal mechanisms, should it be discarded?

\paragraph{Top-Down Design}
The top-down approach consists in designing a set of overlay components that
interact with existing VIM APIs to avoid modifying VIM codebases.
Avoiding codebase modifications is of particular importance in OpenStack since
a new version is released every six months with a lot of changes in the
codebase, whereas changes rarely impact the APIs. Therefore, a top-down design
makes VIM development and the overlay system components independent. If
designed to do so, the system can easily allow L4 requirements, \ie the
support of different versions and types of VIMs - as demonstrated by 
FogBow~\cite{brasileiro2016fogbow}.

Unfortunately, a top-down design cannot satisfy all L2 requirements
without extending or revising the existing VIM codebase. For example,
OpenStack Tricircle~\cite{tricircle}, a top-down project to allow
virtual networking across different sites, ended up "breaking" the
core of OpenStack by introducing specific L2 mechanisms. Such
intrusive modifications negate the aforementioned independence.
%
Moreover, L2 features in general require to
re-implement many low-level VIM functionalities at the overlay level. For
instance, the OpenStack ``boot a VM'' process looks as follow from a
bird's-eye view: (1) Get the URL of the image by looking up into the
database, (2) Schedule and boot the VM. Thus, booting a VM on
\emph{Site 1} using an image defined in \emph{Site 2} would require to
implement a dedicated workflow at the overlay level in order to
interact with both sites and copy the image from the image manager
(\ie Glance) of \emph{Site 2} to the one of \emph{Site 1} before booting the VM. 
%
This is valid for most L2 features such as a fine-grained authorization management
with different rights in different edge sites.
%
%%FW: I do not like the example above, it already assumes a specific solution, I suggest we use the following
%For instance, when sites are specified implicitly (cf. L2.2) a scheduling functionality 
%needs to be implemented at the overlay-level, similar to the scheduling functionality
%already in \texttt{nova}.  
%This is in fact the case for most L2 features such as a fined authorization management
%with different rights in different Edge sites.
%% Also, I do not know if the following is a pro or a con
%Finally, this top-down overlay should be developed \emph{a la
%  peer-to-peer} to cope with L3 requirements (\ie network split-brain
%issues).



%------Moreover, if different VIM flavors are deployed, it might become impractical to provide a single VIM management-view to users as different VIMs might have specific features that need to be managed.


\paragraph{Bottom-Up Design}
In the bottom-up design, the system is not limited by what is
available through VIM APIs. If one can make OpenStack natively
collaborative, features can be supported across the entire
edge infrastructure for free. For instance, the aforementioned
OpenStack ``boot a VM on Site 1 with image on Site 2'' process would
be feasible without modifying the VIM codebase if \emph{Site 1} can
either directly contact the image manager of \emph{Site 2}, or share
the database backend with \emph{Site 2}.

However, the efficiency and ease of use brought by collaboration comes at a
cost. Since OpenStack has not been designed to be collaborative, most
mechanisms must be revised to consider possible side-effects related to
collaboration operations.
%, like one of the edge scenarios, in mind.
%
For instance, a VM boot process initiated on \emph{Site 1} can finally be
completed on \emph{Site 2}. The question is then to define where the states
related to this VM should be stored, keeping in mind the split-brain
challenge.

Finally, L4 requirements cannot be intrinsically satisfied by the
bottom-up approach while L5 implies strong limitations regarding how
collaborations should be implemented (for instance sharing internal
states of different Edge sites between operators looks unlikely).

% This kind of collaboration may be achieved with a collaborative
% backend that relies on a notion of \emph{space}. \emph{Space} says in
% which space the OpenStack lookup and write data. It could be an Edge
% site or a combination of Edge sites if the collaboration is needed. It
% leads to a more efficient system, one without the need to repeatedly
% implement features at different levels.  However, implementing a
% backend based on a notion of space represents a scientific challenge.
%

% With the high industry demand
% for a working Edge system, this means that modifying or extending
% OpenStack is in most cases too expensive (time and resource wise).
% Note also that such backend will not free the developer to implement
% specific workflows for intra-services scenarios\RACm{For instance the
%   ``list of actually scheduling VM'' could not be reused from vanilla
%   code}.


\paragraph{Summary}
In the end, there are \emph{pros} and \emph{cons} for both approaches,
and none individually seems to meet all technical and business
requirements of edge scenarios.
On the technical perspective, the bottom-up
design seems to be the most appropriate to cope with L1, L2 and L3
requirements, while the top-down strategy is the only one to satisfy L4
and L5.
%
On a business perspective, a top-down approach can easily fulfill the
short and medium-term Time-To-Market (TTM) requirements. In that
sense, top-down is the working solution people consider first, and is
in fact the one the open-source community has been working towards so
far. However, our community should investigate a long-term solution
that lays on a bottom-up approach, tackling general scientific
challenges such as the definition of a reference architecture for a
resource management system for Edge infrastructures, and more specific
ones like sharing internal states in an Edge context.







%%As a starting point, early \emph{bottom-up} efforts
%% \JM{Ronan, reference to your paper?}
%%  can be taken into account along with mechanisms that have been implemented in \emph{top-down} solutions but can be applied in a \emph{bottom-up} way~\cite{ericsson-p2p}.


%% \JM{Would a picture somewhere in the paper with a simplified view of a top-down system vs a bottom-up one help the reader? Maybe in section 3.3? I think it would be useful, but we would need to cut somewhere to make it fit in the 5 pages}

%NOTES:
%The way to go is bottom up. Top-down, fastest TTM, a quick fix. That is how everyone is doing it, and thatis the way. BUt that is no the way it should be designed. Short-medium term, top-down...long-run, bottom-up.
%Refer to:
%http://ieeexplore.ieee.org/abstract/document/1334777/
%Top-down: requirements, then implement it
%Bottom-up: not all requirments, but implement something. Can top-down approaches be partially used and implemented in a bottom-up.