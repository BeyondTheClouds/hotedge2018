\section{Conclusion}
\label{sec:conclusion}

The emergence of Network Function Virtualization (NFV) technologies as
well as Internet of Things (IoT) and Augmented/Virtual Reality (AR/VR)
applications herald a new era of Cloud Computing where infrastructures
need to leverage resources at the edge of the network in order to cope
with latency requirements.  Despite this growing need, there is no
cloud management system that is designed with such an infrastructure
in mind.

In this paper, we presented reflections to initiate discussions on this
hot topic. 
%[Dim] I'm not sure this sentence is correct:
We outlined a systematic grouping of the features that administrators/DevOps
expect and the requirements an edge infrastructure puts on a cloud management
system.
%e also identified existing efforts in this area and which may be used as a starting point to a full-blown solution.
We briefly studied the use of existing IaaS managers (\ie OpenStack)
to control an edge infrastructure highlighting the need for an
effective collaboration between the edge sites.  
We then discussed pros and cons of two possible strategies to follow when
designing a solution: (1) a \emph{top-down} approach which designs overlay
components that interact with underlying IaaS managers without modifying their
codebase; (2) a \emph{bottom-up} approach that extensively
improves existing software to fulfill the requirements. We concluded
that on the short-term, a \emph{top-down} approach is required to have
a timely working solution available. On a long-term, both approaches should be
considered simultaneously: \emph{bottom-up} to realize a native and efficient
system for the edge; \emph{top-down} to enable collaboration between different
cloud stacks (\eg OpenStack, Kubernetes).

Finally, it is noteworthy that our study lies on one kind of edge
infrastructure. Other variants can be envisioned where, for example,
hardware resources provided by third-party users can dynamically join
and leave the infrastructure. For such cases, this study needs to be
extended to integrate additional features not discussed in this paper.

%\AL{We should highlight that we conducted our study on one kind of  edge infrastructure. People start to envision more advanced scenarios where third party entities might provided hardware resources that can join the infrastructure. In such a case, our resource managmeent system will have to integrate additional features not discussed in the paper}
