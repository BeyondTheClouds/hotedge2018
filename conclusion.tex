
\section{Conclusion}
\label{sec:conclusion}

The popularity of IoT, the NFV trend as well as the new breed of applications
like AR/VR, heralds a new era of cloud computing where cloud
platforms need to leverage Edge infrastructures in order to be able to
give an acceptable level of cost and performance to users and
operators.  Despite this growing need however, there does not seem to
be any cloud management system that is designed having such an
infrastructure in mind. 

In this paper, we outline a systematic
grouping of the features that end-users expect and the requirements an
edge infrastructure puts on a cloud management system. 
%e also identified existing efforts in this area and which may be used as a starting point to a full-blown solution.
We study the use of existing IaaS managers (\ie OpenStack) to control an Edge infrastructure, and motivate the need for an  effective collaboration of the resource management system accross the entire Edge.
We then proceed to describe two possible strategies to follow when designing a solution: (1) a
\emph{top-down} approach that keeps existing software untouched and
attempts to implement features by software that runs on the top and (2)
a \emph{bottom-up} approach that extensively improves existing
software in order to fulfill the requirements. We discuss the pros and
cons of each strategy and give our recommendation on how we believe we
should move forward in designing a cloud system for an edge
infrastructure.

Finally, it is important to note that our study lies on one kind of edge infrastructure. Other variants can be envisions where, for example, hardware resources can dynamicaly join and leave the edge infrastructure. For such cases, this study needs to be further extended to integrate additional features not discussed in this paper.

%\AL{We should highlight that we conducted our study on one kind of  edge infrastructure. People start to envision more advanced scenarios where third party entities might provided hardware resources that can join the infrastructure. In such a case, our resource managmeent system will have to integrate additional features not discussed in the paper}
