
\section{Conclusion}
\label{sec:conclusion}

The emergence of Network Function Virtualization (NFV) technologies as
well as Internet of Things (IoT) and augmented/virtual reality (AR/VR)
applications herald a new era of Cloud Computing where infrastructures
need to leverage resources at the edge of the network in order to cope
with latency requirements.  Despite this growing need, there is no
Cloud management system that is designed with such an infrastructure
in mind.

In this paper, we presented reflections to initiate discussions on this
hot topic. We outlined a systematic grouping of the features that
administrators/DevOps expect and the requirements an edge
infrastructure puts on a cloud management system.
%e also identified existing efforts in this area and which may be used as a starting point to a full-blown solution.
We briefly studied the use of existing IaaS managers (\ie OpenStack)
to control an Edge infrastructure to illustrate the need for an
effective collaboration between the Edge sites.  We then discussed
pros and cons of two possible strategies to follow when designing a
solution: (1) a \emph{top-down} approach that keeps existing software
untouched and attempts to implement features by software that runs on
the top and (2) a \emph{bottom-up} approach that extensively improves
existing software in order to fulfill the requirements. We concluded
that from the long-term viewpoint, both approaches should be adressed
simulatneously. The former to address collaborations between different stacks and operators leveraging contributions made by the latter.
%\AL{Last sentence to be fixed}

Finally, it is noteworthy that our study lies on one kind of edge
infrastructure. Other variants can be envision where, for example,
hardware resources provided by third-party users can dynamicaly join
and leave the infrastructure. For such cases, this study needs to be
extended to integrate additional features not discussed in this paper.

%\AL{We should highlight that we conducted our study on one kind of  edge infrastructure. People start to envision more advanced scenarios where third party entities might provided hardware resources that can join the infrastructure. In such a case, our resource managmeent system will have to integrate additional features not discussed in the paper}
