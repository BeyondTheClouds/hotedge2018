
\section{Conclusion}
\label{sec:conclusion}

The popularity of the Internet of Things, the current trend of
virtualizing network functions as well as a new breed of applications
such as AR/VR, heralds a new era of cloud computing where cloud
platforms need to leverage edge infrastructures in order to be able to
give an acceptable level of cost and performance to users and
operators.  Despite this growing need however, there does not seem to
be any cloud management platform that is designed having such an
infrastructure in mind. In this paper, we outlined a systematic
grouping of the features that end-users expect and the requirements an
edge infrastructure puts on a cloud management system. We also
identified existing efforts in this area and which may be used as a
starting point to a full-blown solution. We then proceed to describe
two possible strategies to follow when designing a solution: (1) a
\emph{top-down} approach that keeps existing software untouched and
attempts to implement features by software that run on the top and (2)
a \emph{bottom-up} approach that extensively improves existing
software in order to fulfill the requirements. We discuss the pros and
cons of each strategy and give our recommendation on how we believe we
should move forward in a design for a cloud platform for an edge
infrastructure.


\AL{We should highlight that we conducted our study on one kind of
  edge infrastructure. People start to envision more advanced
  scenarios where third party entities might provided hardware
  resources that can join the infrastructure. In such a case, our
  resource managmeent system will have to integrate additional
  features not discussed in the paper}
