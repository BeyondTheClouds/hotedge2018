% TEMPLATE for Usenix papers, specifically to meet requirements of
%  USENIX '05
% originally a template for producing IEEE-format articles using LaTeX.
%   written by Matthew Ward, CS Department, Worcester Polytechnic Institute.
% adapted by David Beazley for his excellent SWIG paper in Proceedings,
%   Tcl 96
% turned into a smartass generic template by De Clarke, with thanks to
%   both the above pioneers
% use at your own risk.  Complaints to /dev/null.
% make it two column with no page numbering, default is 10 point

% Munged by Fred Douglis <douglis@research.att.com> 10/97 to separate
% the .sty file from the LaTeX source template, so that people can
% more easily include the .sty file into an existing document.  Also
% changed to more closely follow the style guidelines as represented
% by the Word sample file. 

% Note that since 2010, USENIX does not require endnotes. If you want
% foot of page notes, don't include the endnotes package in the 
% usepackage command, below.

% This version uses the latex2e styles, not the very ancient 2.09 stuff.
\documentclass[letterpaper,twocolumn,10pt]{article}
\usepackage{usenix,epsfig,endnotes}

%%%%%%%%%%%%%%%%%%%%%%%%%%%%%%%%%%%%
% Accents français
%%%%%%%%%%%%%%%%%%%%%%%%%%%%%%%%%%%%
\usepackage[utf8]{inputenc}
\usepackage[american]{babel}
\usepackage[T1]{fontenc}

\usepackage{xspace}
\usepackage{hyperref}
\usepackage[capitalise, noabbrev]{cleveref}

%%%%%%%%%%%%%%%%%%%%%%%%%%%%%%%%%%%%
% Todo notes
%%%%%%%%%%%%%%%%%%%%%%%%%%%%%%%%%%%%
\usepackage[textwidth=17mm]{todonotes}
\newcommand{\customtodo}[4]{
        \todo[color=#2,inline,size=\small]{
                \ifx&#3&
                        \textbf{#1} #4
                \else
                        \textbf{#1$\Rightarrow$#3} #4
                \fi
        }
}
\newcommand{\AL}[2][]{\customtodo{AL}{green!50}{#1}{#2}}
\newcommand{\ALm}[1]{\todo[color=green!50,size=\small]{#1}}
\newcommand{\JM}[2][]{\customtodo{JM}{red!20}{#1}{#2}}
\newcommand{\JMm}[1]{\todo[color=red!15,size=\small]{#1}}
\newcommand{\FW}[2][]{\customtodo{FW}{brown!20}{#1}{#2}}
\newcommand{\RAC}[2][]{\customtodo{RAC}{yellow!15}{#1}{#2}}
\newcommand{\RACm}[1]{\todo[color=yellow!15,size=\small]{#1}}
\newcommand{\DPm}[1]{\todo[color=yellow!15,size=\small]{#1}}
\newcommand{\DP}[2][]{\customtodo{DP}{yellow!15}{#1}{#2}}


\newcommand{\ie}[0]{{\em i.e.},\xspace}
\newcommand{\vs}[0]{{\em vs.}\xspace}
\newcommand{\eg}[0]{{\em e.g.},\xspace}
\newcommand{\etal}[0]{{\em et al.}\xspace}
\newcommand{\wrt}[0]{{\em w.r.t.}\xspace}
\newcommand{\aka}[0]{{\em a.k.a.}\xspace}
\newcommand{\via}[0]{{\em via}\xspace}

%%%%%%%%%%%%%%%%%%%%%%%%%
% Table add-ons
%\usepackage{array,multirow}
\usepackage{longtable, booktabs, multirow}
\usepackage{array,booktabs}
\newcolumntype{L}{@{}>{\kern\tabcolsep}l<{\kern\tabcolsep}}
\usepackage{colortbl}

\newcolumntype{P}[1]{>{\centering\arraybackslash}p{#1}}
\newcolumntype{M}[1]{>{\centering\arraybackslash}m{#1}}

\newcommand{\Tstrut}{\rule{0pt}{2.75ex}\rule[-1.5ex]{0pt}{0pt}}
\newcommand{\Bstrut}{\rule{0pt}{1.5ex}\rule[-1.5ex]{0pt}{0pt}}
\newcommand{\Rstrut}{\rule{0pt}{1.75ex}\rule[-1.25ex]{0pt}{0pt}}

\newcommand{\thickhline}{\noalign{\hrule height 1.5pt}}

\sloppy

\usepackage{url}
\usepackage{paralist}
\usepackage[loose]{subfigure}

% listing preference
\usepackage[scaled=0.85]{beramono}
\usepackage{listings}
\lstset{numbers=none,basicstyle=\ttfamily}
\usepackage{import}


\usepackage{enumitem}


\begin{document}

%don't want date printed
\date{}

%make title bold and 14 pt font (Latex default is non-bold, 16 pt)
\title{\Large Edge Computing Resource Management System: a Missing Critical Building Block!}

%for single author (just remove % characters)
\author{
{\rm Ronan-Alexandre Cherrueau, Adrien Lebre, Dimitri Pertin}\\
STACK Research Group - IMT-Atlantique, Inria, LS2N, France
\and
{\rm Fetahi Wuhib and  Joao Monteiro Soares}\\
Ericsson, Sweden\\
% copy the following lines to add more authors
% \and
% {\rm Name}\\
%Name Institution
} % end author

\maketitle

% Use the following at camera-ready time to suppress page numbers.
% Comment it out when you first submit the paper for review.
\thispagestyle{empty}


\subsection*{Abstract}
Driven by several use-cases related to IoT and NFV, a cloud computing platform that leverages edge infrastructures has currently garnered a lot of attention. Despite this attention however, a management platform that is suitable for such an infrastructure is yet to be developed. In this paper, we introduce requirements that edge infrastructures and their users put on a cloud management system. We identify two approaches, top-down and bottom-up, for building a cloud management that fulfills the requirements. We discuss the pros and cons of each approach, and finally give our recommendation on which approach to follow.         

\section{Introduction}
\label{sec:intro}


% Edge Infrastructures are the next platform
With the emergence of Internet of Things applications (IoT) and new
usages related Network Function
Virtualization technologies (NFV), Cloud and Network communities are now advocating
for going towards massively distributed small sized infrastructures
that are deployed at the edge of the network.
%
Referred to as the Edge Computing paradigm, this model has been attracting
growing interest for the last few years as it enables delivering Cloud
Computing capabilities closer to end-users, their related devices, and
applications.
%

%% A lot of academics works focus on how to leverage these
%% infrastructures to address specicifc use-cases
While several academic studies have highlighed the advantages of Edge
Computing infrastructures in different
domains~\cite{bonomi2012fog,zhang2015cloud,7574435,satyanarayanan2017emergence},
the development of
%%  We need a resource management system that enables administrators to
%% operate and end-users to use edge resources.
an appropriate management system that will enable, on the first hand,
an operator to aggregate, supervise and expose such massively
distributed resources and, on the other hand, developers to implement
new kinds of services that may be deployed and managed on demand is
still an open question.

%%  Delivierng en Edge IaaS system is complex
To the best of our knowledge, only a few activities have been
investigating how such a system can be designed overall.  With their
IOx and Fog Director software solutions~\cite{bonomi2012fog}, Cisco allows IoT
applications to run on infrastructures composed of NFV-enabled
hardware (at the edge) and existing centralized clouds. However, their
solution does not allow to run Virtual Machines  and does not compete
with existing IaaS platforms.
%
Although Edge infrastructures significantly differ from traditional Cloud ones
regarding heterogeneity, dynamicity and the potential massive
distribution of resources and networking environment, administrators and end-users expect the same capabilities.
To mitigate the complexity of a complete stack, most advanced proposals have been focusing
on orchestrating applications accross distinct independent cloud
systems (\aka virtual infrastructure managers).
%
Among the major projects, one can cite the architecture
defined by the European Telecommunications Standards Institute (ETSI)
MEC Industry Specification Group~\cite{7574435} or more recently the
ONAP framework~\cite{onap}, which is supported by several industrial groups. 
From the academic side, brokering approaches have been also proposed.
For instance, the FogBow proposal~\cite{brasileiro2016fogbow} aims to support large
federations of Infrastructure-as-a-service (IaaS) cloud providers.

While all these projects have extended current interfaces (APIs) to
turn location into a first-class citizen, and allow developpers to
manage edge application life cycle in geo-distributed manner, they do not
provide a well-suited software stack to operate and use edge infrastructures like cloud computing ones. 

This paper proposes several
contributions to make progress on the aformentioned question.
\begin{itemize}
\item A list of expected features from bothe the administrators and end-users' perspectives;
\item A discussion of pros/cons of top/down 
\end{itemize}

The remaining of the paper ...



% In this paper, we provide a list of expected features for both administrators and admins.
% We investigate whether a stack such as OpenStack, the defacto open-source standard can fullfil these requirements
% We finally discusss two possibles ways for moving forward: top/down vs bottom/up.


% TODO: talk about:
% The NIST Cloud Computing Program (NCCP) recognized, in the USG Cloud Computing Standards and Technology Roadmap (NIST SP 500-293), the importance of advancement and development of frameworks to support seamless implementation of federated community cloud environments. In August 2017, the NCCP Public Working Group (PWG) on Federated Cloud (PWGFC) initiated a collaborative effort with the IEEE P2302 group, Standards for Intercloud Interoperability and Federation (SIIF), to advance Federated Cloud through the development of a conceptual architecture and a vocabulary. The IEEE P2302 will develop a standard based on the NIST model.  Additional information can be found at https://collaborate.nist.gov/twiki-cloud-computing/bin/view/CloudComputing/FederatedCloudPWGFC  



\section{Administrators/Developers' Requirements}
\label{sec:requirements}

% \begin{figure}[t]
%   \centering
%   \def\svgwidth{\columnwidth}
%   \input{figures/sites.pdf_tex}
%   \caption{Partial representation of the targeted edge infrastructure, composed
%     of multiple sites. Each site is composed of many servers (represented by
%     black bullets) and have a set of users (depicted by black squares). The red
%     dashed line depicts a split-brain situation that separates the
%     infrastructure into $2$ partitions, isolating \emph{Site 1} from \emph{Site
%     2} and \emph{3}.}
%   \label{fig:sites}
% \end{figure}


\begin{table*}
    \centering
        
\scriptsize
\begin{tabular}{@{} l L L L @{} >{\kern\tabcolsep}l @{}}
    \toprule
  %\begin{tabular}{@{} p{.25\textwidth} p{.25\textwidth}p{.25\textwidth}p{.25\textwidth} @{} >{\kern\tabcolsep}l @{}}    \toprule

    \emph{Levels} & \emph{Admin}& \emph{User} & \emph{Both} \\
    \midrule

    L1: Operate/use any site &
    Manage any site: install, upgrade site's&
    Provision compute, storage, network&
    Collect metrics and ensure security,\\ 

    &
    services; manage users, flavors, quotas&
    resources on-demand on any site &
    integrity and resiliency for any site\\
    %\\% add space between levels

    \rowcolor{black!20}[0pt][0pt]
    L2: Operate/use several sites &
    Manage multiple sites simultaneously&
    Cross-site collaborative resources&
    L1 but over a set of sites\\

    \rowcolor{black!20}[0pt][0pt]
    - L2.1 Explicit manner:&
    Manage a specific set of sites &
    Provision on a specific set of sites &
    Aggregated metrics from multiple sites\\
    
    \rowcolor{black!20}[0pt][0pt]
    - L2.2 Implicit manner:&
    Cross-site autonomous management&
    Cross-site autonomous provisioning&
    and collaborative security mechanisms\\
    %\\% add space between levels

    L3: Robustness \wrt split brains &
    &
    &
    L1 for an isolated site; L1 and L2 for\\


    - L3.1 Application robustness:&
    \hfill ------ \hfill &
    Access reachable applications&
    isolated sets of sites\\

    - L3.2 Management service robustness:&
    Manage reachable site(s) &
    Provision on reacheable site(s)&
    Support intermittent connectivity\\
    %\\% add space between levels

    \rowcolor{black!20}[0pt][0pt]
    L4: Multiple Cloud environments &
    &
    &
    L3 with different IaaS environments\\

    \rowcolor{black!20}[0pt][0pt]

    \rowcolor{black!20}[0pt][0pt]
    - L4.1 Different IaaS versions:&
    Manage different IaaS versions&
    Provision on different IaaS versions&
    Discover sites' capabilities and\\
    
    \rowcolor{black!20}[0pt][0pt]
    - L4.2 Different IaaS technologies:&
    Manage different IaaS technos&
    Provision on different IaaS technos&
    compatibility\\
    %\\% add space between levels

    L5: Multiple operators &
    \hfill ------ \hfill &
    Provision on one or many sites&
    L4 with multiple operators\\ 

    \bottomrule

\end{tabular}


    \caption{Classification of the requirements to administrate and use edge
    computing infrastructures in $5$ levels.}
    \label{tab:requirements}
\end{table*}


As previously mentioned, administrators and end-users of edge
infrastructures expect to get a set of high level mechanisms whose
assembly results in a system capable of operating and using a
geo-distributed IaaS infrastructure (Compute, Storage and Network
managers, monitoring and administrative tools)~\cite{moreno2012csp}.
However, considering each of these services independently is not
sufficient enough to identify the challenges our community should deal with.
%
In this section, we consider and classify concrete actions administrators and
developers/end-users expect to perform on edge computing infrastructures.
\AL{Polish (in particular by adding a transition)}.
The classification is based on $5$ \emph{levels},
starting from the easiest aspects, \ie interacting with a single site
-- level 1 (L1) to more complex aspects like managing multiple
sites (L2), up to considering they can be owned by different operators
(L5).  \Cref{tab:requirements} summarizes the classification we detail
in the following:

\paragraph{L1: Operate/use any site}
This level contains all the requirements at the scope of a single site of the
infrastructure. The second row of \cref{tab:requirements} lists those
requirements expected from administrators and users. To operate a single site,
like \emph{Site 1} depicted in \cref{fig:sites}, the resource management
services defined previously must be installed (or upgraded) by an
administrator. After what, admins can use those services to manage users,
accesses, flavors (\ie available capacities of compute resources) or quotas,
regarding \emph{Site 1}.
% edge sites are remote & unmanned and should administrated remotely
% tools need to support intermittent network access to the site

Any end-user of the infrastructure expects to be able to provision compute,
network or storage resources on any single site, supposing he is authorized to.
For instance, a user located at \emph{Site 1} should be able to boot a VM in
\emph{Site 2}, using images, flavors or networks defined in \emph{Site 2} -- as
long as the site is reachable.

Furthermore, administrators and users expect metrics to be monitored (\eg
related to resources, users, and projects) from any single site. This is used
for instance for respectively managing quotas and listing existing resources.
As mentioned, Edge infrastructures are dynamic, \ie sites can join and leave the
infrastructure (on purpose or due to intermittent network access). As a
consequence, tools also need to support this churn effect.
Regarding the security aspect, one requirement expected by both admins and
users is that the communications in the infrastructure are secured (\eg
isolated, encrypted) and that actions are logged for auditing.

\paragraph{L2: Operate/use several sites:}
We now consider in this level that the operations previously defined in L1
cover at least two sites. For instance, an administrator might desire to
configure users access on a per-site basis. Regarding users, they also expect
a collaboration between sites to boot, for instance, a VM on \emph{Site 1},
using an image defined in \emph{Site 2}. To that end both administrators and
users require to collect data that belong to several sites. We define here two
sub-levels as depicted in \cref{tab:requirements}. Such collaboration between
sites can be either explicit (\ie the targeted sites are explicitly specified),
referenced as the sub-level L2.1, or implicit (L2.2). The implicit manner
suggests that policies (\eg performance objectives, energy consumption) and
constraints (\eg affinity, underlying hardware) are provided by admins and users so
that an edge-aware orchestrator takes the right decisions regarding the users'
desiderata and the state of the infrastructure (\eg auto-scaling, relocating
workloads between sites, re-scheduling faulty resources across sites).
% should be elaborated and/or refs should be given here

\paragraph{L3: Robustness wrt split brains:}
In this level, we now consider split brains which correspond to the situation
where the infrastructure is partitioned due to communication failures.
\Cref{fig:sites} depicts such situation where \emph{Site 1} is isolated from
\emph{Site 2} and \emph{3}. In such case, all operations defined in L1 must be
guaranteed for a partition composed of a single site like \emph{Site 1}
(supposing it is reachable). For instance, a user in \emph{Site 1} should be
able to use \emph{Site 1}'s resources despite the split brain. Regarding
partitions containing several sites, L1 and L2 operations must be guaranteed
for any set of sites inside the partition.
% what about quotas? how to manage them in case of split-brain?
% suppose a user has a global quotas on several sites, it is impossible to
% determine its global consumption in case of split-brains?
Two sub-levels are proposed here: L3.1 considers the robustness of already
deployed resources (\eg a user should be able to access a deployed application
despite the split brain, if he can reach the related site), while the
robustness of management services, treated in L3.2, is required for both admins
and users.
% tricky issue: management of split brain between collaborative services in
% different sites

\paragraph{L4: Multiple Cloud environments:}
Since edge infrastructures are dynamic infrastructures where sites can join and
leave, different versions of an IaaS manager can exist on different sites. One
requirement here is to consider L3 requirements between sites operated by
different manager versions (L4.1). Another requirement is to prevent vendor
lock-in between multiple generic IaaS manager technologies (\eg OpenStack,
CloudStack), and container orchestrators like Kubernetes (L4.2).
As a consequence, it is necessary to discover site's capabilities to determine
their compatibilities. For instance, it is not possible to migrate a VM from
\emph{Site 1} to \emph{Site 2} if the latter is managed by a container
orchestrator.

\paragraph{L5: Multiple operators:}
We now consider that sites can be owned by different operators. As depicted in
\cref{tab:requirements}, no requirements is expected by administrators since an
operator should not be able to administrate another operator's sites. However,
operators should be able to collaborate to offer their resources to users.
Specific metrics should be collected at the scope of operators to manage for
instance billing between them.


%%% Local Variables:
%%% mode: latex
%%% TeX-master: t
%%% End:

%\section{Design Considerations}
%\label{design_considerations}

\section{System Design}
\label{sec:system_design_considerations}
As previously mentioned, the starting point for this study is an edge
infrastructure composed of hundreds of micro DCs (\aka edge sites)
operated by the OpenStack software suite.
%
OpenStack is composed of two kinds
of nodes: on the one hand, the compute/storage/network nodes are
dedicated to deliver the XaaS capabilities, such as hosting VMs (\ie,
data plane); on the other hand, the control nodes are in charge of
executing the OpenStack services (\ie, control plane).
%
Although each edge site provides enough resources to host control and
compute capabiliities, several deployments scenarios can be
envisioned, including one global or several OpenStack instances. In
this section, we discuss two particular scenarios. The first one
consists in deploying all control services on one site of the
infrastructure and considering remote resources as compute
nodes.  The second one corresponds to the guidelines presented on the
OpenStack website to supervise a multi-regions infrastructure.
%
This preliminary discussion, which considers two extreme deployment
scenarios, enables us to justify why collaborations is mandatory to
fulfill the requirements identified in Section~\ref{sec:requirements}.
%
We emphasize that although OpenStack has been considered, alternative
systems such as Kubernetes would have been lead to the same conclusion
in terms of resource management and collaboration needs.  \AL{Put
  references Kubernetes}

\subsection{Centralized Management and Remote Compute/Storage/Network Nodes}
\label{subsec:centralized_os}
At coarse-grained, the first scenario consists in operating an edge
infrastructure like a traditional cloud computing one.  The ``only''
difference is related to the wide-area network links between the
control services and the compute nodes.  Futhermore, it might be
possible to define multiple availability zones in order to make a
distinction between each geographical location.
%
Technically speaking, OpenStack services are organized following the Shared Nothing
principle. Each instance of a service (i.e., service worker) is
exposed through an API accessible through a Remote Procedure Call
(RPC) system implemented, on top of a messaging queue or via web
services (REST). This enables a weak coupling between services and
thus a large number of deployment possibilities such as this one.
%
øFrom the requirements viewpoint (see Section~\ref{sec:requirement}) L1
and most of the L2 features~\footnote{Because the infrastructure can
  be spread over several network domains, advanced network operations
  cannot be satisfied without any change in the OpenStack codebase.}
can be fulfilled in a straightforward way.  However, in addition to
the scalability limitation that will prevent such a deployment of
managing thousands of compute nodes, it is not possible to satisfy the
L3 expectations.  Even if the organization of the OpenStack services
respects the Shared Nothing principle, most services create and
manipulate logical objects that are persisted in shared
databases. While this enables service workers to easily collaborate,
it imposes a permanent connectiviy between compute nodes, services and
their DBs. In other ways, this scenario provides a Single Pane of
Glass for the administrator viewpoint but at the same time is a Single
Point of Failure).\AL{Fix this last sentence}


\subsection{Segregate into Regions}


\subsection{Effective Collaboration is Needed}
Each Edge Site
provides control and compute capabilities (see~\ref{sec:requirements})
on top of dozens of servers. They \emph{collaborate} with each-other
to shape up the Edge Infrastructure and a Wide Area Network (WAN)
interconnects them through both wired and wireless network links. This
results in up to 300ms of RTT between two Edge Sites and common
disconnections.

Despite the fallibility of the network, and frequent isolation risks
of an Edge Site from the rest of the Infrastructure (\ie network
split-brain), the Edge Infrastructure may be kept sustainable. This is
achieved by supposing a collaboration a la peer-to-peer, that is, an
Edge Site always serves local resources and collaborates with other
Edge Sites if needed. Thus, a user can make a SSH on his VM during a
network split-brain if he gets a local access to the Edge Site. The
same user can also start a VM using the VMI of an other Edge Site if
the collaboration can be established.

When it comes to develop this resource management system for the Edge
computing, the developer has two fundamental design options: a \emph{top-down} or
\emph{bottom-up}. Both designs impact how to handle the
collaboration needed by such a system.

\paragraph{Top-Down Collaboration}
Design option that implements the collaboration required by
federating IaaS managers' API. In other words, it leverages on existing IaaS platforms as they are made available today without introducing modifications/extensions. Examples of approaches following a \emph{top-down} design are: ONAP~\cite{onap}, Kingbird~\cite{kingbird}, FogBow~\cite{brasileiro2016fogbow} and p2p-OpenStack~\cite{ericsson-p2p}


\paragraph{Bottom-Up Collaboration} Design option that lays on making IaaS mechanisms/services directly collaborative. For example, having two OpenStack Nova services able to cooperate and communicate directly would be a realization of a \emph{bottom-up} design. Such design option implies either the modification/extension of existing IaaS platforms or the creation of a completely new system. Examples of approaches following a \emph{bottom-up} design are:
\AL[Comment by Joao]{Any reference to existing approaches? Guys, do you have, for example, references to your keystone shared db work? Could cells be considered as an example? }

On the one hand, \emph{top-down} options have been the most explored, but can these alone fulfil all expected requirements? On the other hand, \emph{bottom-up} options seem to disruptive the design principles of most existing IaaS platform, but is that in fact the case?



%%% Local Variables:
%%% mode: latex
%%% TeX-master: t
%%% End:

\section{Design Discussions}
\label{sec:design_discussion}
\AL[Fetahi/Joao]{Please start the discussion between top/down and bottom/up approaches}

There are two fundamental ways to address the design of the system: 1) top-down, leveraging on existing VIMs as they are made available today (i.e. without modifications/extensions); 2) bottom-up, modifying/extending existing VIMs or even creating a completely new system. 

In a top-down case, a system  would rely on existing VIM APIs and no modifications to the base code of a VIM would be made. For example, OpenStack provides a new release every six months and the code varies a lot between each release, however changes rarely impact the APIs. Such approach would give full independency to the development of VIMs as well as to remaining edge system components. Moreover, if designed to do so, the system can easily allow to mix different VIMs (e.g. OpenStack and OpenNebula) and different versions of a VIM (e.g. OpenStack Pike and OpenStack Queens release). 
However, if one wants to operate a complete edge infrastructure through a single entry point, it means that certain VIM low-level services need to be reimplemented at a higher-level (e.g. authentication, resource scheduling, monitoring, etc). Moreover, if the latter case applies and if multiple VIMs are deployed, it might become impractical to provide a single VIM management-view to users as different VIMs might have specific features that need to be managed. It is also most certain that in this approach, a reduced/limited set of functionalities will be available compared to a single VIM instance functionalities. In other words, certain requirements cannot be fulfilled without changing the natice VIM (e.g. live migration across sites).


In a bottom-up case, the system is not limited by existing (VIM) features. This can allow the creation of a more efficient systems, for example, without the need to repeatedly implement features at different levels. Moreover, it can potentially fullfil any and all (reasonable) requirements since any feature missing can be implemented. 
However, efficiency and freedom come at a cost. Writting a completely new system will be expensive (time and resource wise). Equally, extending/modifying and existing system as similar cost. Moreover, a bottom-up approach would certainly mean the entire edge infrastructure would rely on a specific platform/VIM. 


The above two approaches each have their pros and cons, and none individually can fullfil all requirements of an edge infrastructure. On the one hand, a top-down approach brings modularity and a certain level of independence between components, however it is limited by that same independence as it requires low-level features to be enabled. On the other hand, a bottom-up approach can be very costly without in the end being able to provide all required features. It becomes somehow easy to state that a system for the edge requires a two-side approach, where top-down bottom-up approaches meet. The real question becomes then, where shall these meet?
\AL[Comment by Joao]{After reading this first draft, everything seems so obvious. Where is the value here? Can we further elaborate on the last question? i.e. where should the two approaches meet? Maybe we do not want to make a hard conclusion, but we can point that the community has not yet settled on one way and refer to the related work. But again, this would still leave the problem open...and again, would such statements value enough for the paper? 
	
	I am having at this point a hard time on finding our "punch line". What is the idea that we want to convey in this paper on which we want to get feedback? Because in the end we should have an idea over which we want feedback on - since HotEgde is about not well cooked ideas that want feedback}

% 
\section{Gap Analysis}
\label{sec:ga}


\subsection{OpenStack official releases}
\label{subsec:ga_os}




\subsection{OpenStack ongoing extensions}
\label{subsec:ga_extensions}




\subsection{Non-OpenStack approaches}
\label{subsec:non-os}

\subsubsection{ONAP}

The Open Network Automation Platform (ONAP) \cite{web:onap} is a framework to
design and instantiate Virtual Network Functions (VNFs) on top of Virtual
Infrastructure Managers (VIMs), like OpenStack, Microsoft Azure or Amazon Web
Services. Driven by the Linux Foundation since 2017, this active project is
built on a modular architecture where each module is in charge of the designing
or the orchestration of VNFs. For instance, some modules are in charge of
provisioning compute and network resources, while others are in charge of
scheduling, authentication or storing images.

%
%For instance, the Master Service Orchestration (MSO) module coordinates the
%actions to deploy VNFs, while the Data Collector, Analytics and Events (DCAE)
%module is in charge of collecting metrics and raises alarms to other modules.
%

More particularly, ONAP relies on the multi-VIM module to orchestrate multiple
independent infrastructure managers. This module provides an abstraction of the
underlying VIMs and acts as a request broker between ONAP's modules and VIMs. To
that end, it is itself composed of sub-modules to manage inter and intra-VIM
actions (e.g. inter-VIM scheduling, network overlays).



\subsection{Academic proposals}
\label{subsec:academic}





% \input{discussions.tex}

\section{Conclusion}
\label{sec:conclusion}

The popularity of IoT, the NFV trend as well as the new breed of applications
like AR/VR, heralds a new era of cloud computing where cloud
platforms need to leverage Edge infrastructures in order to be able to
give an acceptable level of cost and performance to users and
operators.  Despite this growing need however, there does not seem to
be any cloud management system that is designed having such an
infrastructure in mind. 

In this paper, we outline a systematic
grouping of the features that end-users expect and the requirements an
edge infrastructure puts on a cloud management system. 
%e also identified existing efforts in this area and which may be used as a starting point to a full-blown solution.
We study the use of existing IaaS managers (\ie OpenStack) to control an Edge infrastructure, and motivate the need for an  effective collaboration of the resource management system accross the entire Edge.
We then proceed to describe two possible strategies to follow when designing a solution: (1) a
\emph{top-down} approach that keeps existing software untouched and
attempts to implement features by software that runs on the top and (2)
a \emph{bottom-up} approach that extensively improves existing
software in order to fulfill the requirements. We discuss the pros and
cons of each strategy and give our recommendation on how we believe we
should move forward in designing a cloud system for an edge
infrastructure.

Finally, it is important to note that our study lies on one kind of edge infrastructure. Other variants can be envisions where, for example, hardware resources can dynamicaly join and leave the edge infrastructure. For such cases, this study needs to be further extended to integrate additional features not discussed in this paper.

%\AL{We should highlight that we conducted our study on one kind of  edge infrastructure. People start to envision more advanced scenarios where third party entities might provided hardware resources that can join the infrastructure. In such a case, our resource managmeent system will have to integrate additional features not discussed in the paper}



\bibliographystyle{acm}
{\footnotesize 
\bibliography{hotedge2018}}

\theendnotes

\end{document}







