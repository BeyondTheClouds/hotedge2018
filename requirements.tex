
\section{Admin/Devops' Requirements}
\label{sec:requirements}

% \begin{figure}[t]
%   \centering
%   \def\svgwidth{\columnwidth}
%   \input{figures/sites.pdf_tex}
%   \caption{Partial representation of the targeted edge infrastructure, composed
%     of multiple sites. Each site is composed of many servers (represented by
%     black bullets) and have a set of users (depicted by black squares). The red
%     dashed line depicts a split-brain situation that separates the
%     infrastructure into $2$ partitions, isolating \emph{Site 1} from \emph{Site
%     2} and \emph{3}.}
%   \label{fig:fogedge-archi}
% \end{figure}

\begin{table*}
    \centering
        
\scriptsize
\begin{tabular}{@{} l L L L @{} >{\kern\tabcolsep}l @{}}
    \toprule
  %\begin{tabular}{@{} p{.25\textwidth} p{.25\textwidth}p{.25\textwidth}p{.25\textwidth} @{} >{\kern\tabcolsep}l @{}}    \toprule

    \emph{Levels} & \emph{Admin}& \emph{User} & \emph{Both} \\
    \midrule

    L1: Operate/use any site &
    Manage any site: install, upgrade site's&
    Provision compute, storage, network&
    Collect metrics and ensure security,\\ 

    &
    services; manage users, flavors, quotas&
    resources on-demand on any site &
    integrity and resiliency for any site\\
    %\\% add space between levels

    \rowcolor{black!20}[0pt][0pt]
    L2: Operate/use several sites &
    Manage multiple sites simultaneously&
    Cross-site collaborative resources&
    L1 but over a set of sites\\

    \rowcolor{black!20}[0pt][0pt]
    - L2.1 Explicit manner:&
    Manage a specific set of sites &
    Provision on a specific set of sites &
    Aggregated metrics from multiple sites\\
    
    \rowcolor{black!20}[0pt][0pt]
    - L2.2 Implicit manner:&
    Cross-site autonomous management&
    Cross-site autonomous provisioning&
    and collaborative security mechanisms\\
    %\\% add space between levels

    L3: Robustness \wrt split brains &
    &
    &
    L1 for an isolated site; L1 and L2 for\\


    - L3.1 Application robustness:&
    \hfill ------ \hfill &
    Access reachable applications&
    isolated sets of sites\\

    - L3.2 Management service robustness:&
    Manage reachable site(s) &
    Provision on reacheable site(s)&
    Support intermittent connectivity\\
    %\\% add space between levels

    \rowcolor{black!20}[0pt][0pt]
    L4: Multiple Cloud environments &
    &
    &
    L3 with different IaaS environments\\

    \rowcolor{black!20}[0pt][0pt]

    \rowcolor{black!20}[0pt][0pt]
    - L4.1 Different IaaS versions:&
    Manage different IaaS versions&
    Provision on different IaaS versions&
    Discover sites' capabilities and\\
    
    \rowcolor{black!20}[0pt][0pt]
    - L4.2 Different IaaS technologies:&
    Manage different IaaS technos&
    Provision on different IaaS technos&
    compatibility\\
    %\\% add space between levels

    L5: Multiple operators &
    \hfill ------ \hfill &
    Provision on one or many sites&
    L4 with multiple operators\\ 

    \bottomrule

\end{tabular}


    \caption{Classification of the requirements to administrate and use edge
    computing infrastructures in $5$ levels.}
    \label{tab:requirements}
\end{table*}

In this section, we classify features administrators and devops expect
to find in the context of Edge Computing infrastructures.
Our classification is based on $5$ \emph{levels}, starting from the easiest
aspects, \ie interacting with a single site (considered in level 1 or L1), to
more complex aspects like managing multiple sites (L2), up to considering that
sites can be owned by different operators (L5).
\Cref{tab:requirements} summarizes the classification we detail in the following. 

As previously mentioned, a large part of these features are common to
the ones offered by current IaaS resource management systems. They are
implemented by various services, each of which is in charge of the
management of a particular aspect of the infrastructure
%.In this paper, we consider the Compute, Storage and
%Network managers as well as the monitoring and administrative
%tools
~\cite{moreno2012csp}.
\AL{Not sure whether we should introduce
  service here. Maybe this is something we should only put at the end
  of this section to make a transition with the other ones. The issue
  is that we need it for the moment in L2.}

\paragraph{L1: Operate/use any site}
As depicted in the second row of \cref{tab:requirements}, this level
considers actions both administrators and users expect to perform
when considering a single site, supposing the site is reachable.
%
Most operations are elementary from the Edge viewpoint because they
correspond to the ones already provided by OpenStack for both administrators and devops.  In other words,
each Edge site can be considered as an independent Cloud at this
level. The unmanned aspect only impacts this level by requiring to perform all operations
remotely if needed.
%security
Furthermore, the resource management system should provide means to
ensure the integrity of the hardware resources taking part to the edge
infrastructure. Strategies such as enabling/disabling physical
interactions with the equipment should be considered.

% %
% Regarding administrators,  Administrators should be able to
% install and upgrade the aforementioned resource management
% services. After what, they can use those services to manage
% users, accesses, flavors (\ie available capacities of compute
% resources) and quotas.

% Regarding devops, they should be able to provision compute, network or
% storage resources like any traditional Cloud platform, supposing they
% are authorized to.

% In addition, admins and users share common expectations.
% % collect metrics
% First they want to monitor various metrics related to resources,
% users, and projects . This is used for instance for
% respectively managing quotas and listing existing resources.

\paragraph{L2: Operate/use several sites:}

%% TODO
% For instance, a user of \emph{Site 1} 
In L2, L1 features are considered but over multiple sites (at least
two).  This includes operations such as creating a new resource on a
particular site using resources from another one, managing several
resources or gathering information from various sites simultaneously.
%(interconnecting two VMs with a dedicated network).
Operations can be either intra-services (same service from different sites) or
inter-services (different services from different sites).

%
Concrete operations might consist for instance in configuring users'
access on a per-site basis, listing available VM images or pushing new
ones on multiple sites (intra-service operations)~\ldots From the
devops viewpoint, a user should be able to boot a VM on \emph{Site 1},
using an image defined in \emph{Site 2} (inter-service
operation). Similarly to L1, they both expect metrics collected from
several sites and collaborated mechanisms regarding the security (\eg
secret key sharing, network encryption).

Because collaboration between sites can be either explicit (\ie the
targeted sites are explicitly specified in the operation), or implicit
(\ie the resource management system is in charge of selecting
resources), we have defined two sub-levels as depicted in
\cref{tab:requirements} L2.1 and L2.2 respectively.  The implicit
manner suggests that policies (\eg performance objectives, energy
consumption) and constraints (\eg affinity, underlying hardware) are
provided by admins and users so that the resource management system takes
the right decisions regarding the defined desiderata and the state of
the infrastructure (\eg auto-scaling, relocating workloads between
sites, re-scheduling faulty resources across sites).
% should be elaborated and/or refs should be given here

\paragraph{L3: Robustness \wrt split brains:}
In this level, we consider split brains which correspond to the situation
where the infrastructure is partitioned due to communication failures.
\Cref{fig:fogedge-archi} depicts such situation where \emph{Site 1} can be
isolated from the other sites. In such case, we distinguish two cases: (i) when
a partition is composed of an single site, like \emph{Site 1} in the figure,
all operations defined in L1 must be guaranteed regarding this site, supposing
it is reachable). For instance, a user in \emph{Site 1} must be
able to use \emph{Site 1}'s resources despite the split brain. Regarding
partitions containing several sites, L1 and L2 operations must be guaranteed
for any set of sites inside the partition.
% what about quotas? how to manage them in case of split-brain?
% suppose a user has a global quotas on several sites, it is impossible to
% determine its global consumption in case of split-brains?
Two sub-levels are proposed here: L3.1 considers the robustness of already
deployed resources (\eg a user should be able to access a deployed application
despite the split brain, if he can reach the related site), and is thus limited
to users, while the robustness of management services, treated in L3.2, is
required for both admins and users.
% tricky issue: management of split brain between collaborative services in
% different sites

\paragraph{L4: Multiple Cloud environments:}
Since edge infrastructures are dynamic infrastructures where sites can join and
leave, different versions of an IaaS manager can exist on different sites. One
requirement here is to consider L3 requirements between sites operated by
different manager versions (L4.1). Another requirement is to prevent vendor
lock-in between multiple generic IaaS manager technologies (\eg OpenStack,
CloudStack), and container orchestrators like Kubernetes (L4.2).
As a consequence, it is necessary to discover site's capabilities to determine
their compatibilities. For instance, it is not possible to migrate a VM from
\emph{Site 1} to \emph{Site 2} if the latter is managed by a container
orchestrator.

\paragraph{L5: Multiple operators:}
We now consider L4, plus the possibility that sites can be owned by different
operators. As depicted in \cref{tab:requirements}, no requirements is expected
by administrators since an operator would not share administrative rights to
their sites with other operators. However,
operators should be able to collaborate to offer their sites' resources to
users. Specific metrics should be collected at the scope of operators to
manage for instance billing between them.\\

%footer
Our classification of the requirements expected by both admins and users
highlights that the complexity to develop an edge infrastructure manager
significantly increases at each level. In fact, since L4 and L5 extends L3
respectively with multiple manager environments, and multiple operators, they
can be considered at the same level in our classification (and can be swapped
as a consequence).
Considering now that the challenges are already numerous with L1, L2 and L3, we
propose to drive the system design regarding these levels in the rest of the
paper (leaving L4 and L5 for future works).

