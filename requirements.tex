
\section{Admin/Devops' Requirements}
\label{sec:requirements}

% \begin{figure}[t]
%   \centering
%   \def\svgwidth{\columnwidth}
%   \input{figures/sites.pdf_tex}
%   \caption{Partial representation of the targeted edge infrastructure, composed
%     of multiple sites. Each site is composed of many servers (represented by
%     black bullets) and have a set of users (depicted by black squares). The red
%     dashed line depicts a split-brain situation that separates the
%     infrastructure into $2$ partitions, isolating \emph{Site 1} from \emph{Site
%     2} and \emph{3}.}
%   \label{fig:fogedge-archi}
% \end{figure}

\begin{table*}
    \centering
        
\scriptsize
\begin{tabular}{@{} l L L L @{} >{\kern\tabcolsep}l @{}}
    \toprule
  %\begin{tabular}{@{} p{.25\textwidth} p{.25\textwidth}p{.25\textwidth}p{.25\textwidth} @{} >{\kern\tabcolsep}l @{}}    \toprule

    \emph{Levels} & \emph{Admin}& \emph{User} & \emph{Both} \\
    \midrule

    L1: Operate/use any site &
    Manage any site: &
    Provision on-demand on any site: &
    Collect metrics regarding a single site: \\ 

    (all operations are based in single site) &
    - install services &
    - compute resources&
    Security (secured communications, \\

    &
    - use services (users, flavors, quotas)&
    - network resources&
    auditing, integrity)\\

    &
    - upgrade services&
    - storage resources&
    Resiliency\\
    %\\% add space between levels

    \rowcolor{black!20}[0pt][0pt]
    L2: Operate/use several sites &
    &
    &
    L1 but over a set of sites\\

    \rowcolor{black!20}[0pt][0pt]
    (all operations cover at least two sites) &
    &
    &
    Collect metrics regarding many sites\\

    \rowcolor{black!20}[0pt][0pt]
    - L2.1 Explicit manner&
    Manage a specific set of sites &
    Provision on a specific set of sites &
    \\
    
    \rowcolor{black!20}[0pt][0pt]
    - L2.2 Implicit manner&
    Cross-site Autonomous management&
    Cross-site Autonomous provisioning&
    \\
    %\\% add space between levels

    L3: Robustness \wrt split brains &
    &
    &
    L1 for an isolated site\\ 

    (all operations cover one or many sites) &
    &
    &
    L1 and L2 for an isolated set of sites\\

    - L3.1 Application robustness&
    \hfill ------ \hfill &
    Access user's applications&
    Collect metrics regarding one or many sites\\

    - L3.2 Management service robustness&
    Manage one or many sites &
    Provision on one or many sites&
    \\
    %\\% add space between levels

    \rowcolor{black!20}[0pt][0pt]
    L4: Multiple Cloud environments &
    &
    &
    L3 with different types of IaaS managers\\ 

    \rowcolor{black!20}[0pt][0pt]
    (all operations cover one or many sites) &
    &
    &
    Discover sites' capabilities/compatibilities\\

    \rowcolor{black!20}[0pt][0pt]
    - L4.1 Different IaaS versions&
    Manage different IaaS versions&
    Provision on different IaaS versions&
    \\
    
    \rowcolor{black!20}[0pt][0pt]
    - L4.2 Different IaaS technologies&
    Manage different IaaS technos&
    Provision on different IaaS technos&
    \\
    %\\% add space between levels

    L5: Multiple operators &
    \hfill ------ \hfill &
    Provision on one or many sites&
    L4 with multiple operators\\ 

    (all operations cover one or many sites) &
    &
    &
    Collect metrics from different operators\\
    \bottomrule

\end{tabular}


    \caption{Classification of the requirements to administrate and use edge
    computing infrastructures in $5$ levels.}
    \label{tab:requirements}
\end{table*}

In this section, we classify features administrators and devops expect
to find in the context of Edge Computing infrastructures.
Our classification is based on $5$ \emph{levels}, starting from the easiest
aspects, \ie interacting with a single site (considered in level 1 or L1), to
more complex aspects like managing multiple sites (L2), up to considering that
sites can be owned by different operators (L5).
\Cref{tab:requirements} summarizes the classification we detail in the following. 

As previously mentioned, a large part of these features are common to
the ones offered by current IaaS resource management systems. They are
implemented by various services, each of which is in charge of the
management of a particular aspect of the infrastructure
%.In this paper, we consider the Compute, Storage and
%Network managers as well as the monitoring and administrative
%tools
~\cite{moreno2012csp}.
\AL{Not sure whether we should introduce
  service here. Maybe this is something we should only put at the end
  of this section to make a transition with the other ones. The issue
  is that we need it for the moment in L2.}

\paragraph{L1: Operate/use any site}
As depicted in the second row of \cref{tab:requirements}, this level
considers actions both administrators and users expect to perform
when considering a single site, supposing the site is reachable.
%
Most operations are elementary from the Edge viewpoint because they
correspond to the ones already provided by OpenStack for both administrators and devops.  In other words,
each Edge site can be considered as an independent Cloud at this
level. The unmanned aspect only impacts this level by requiring to perform all operations
remotely if needed.
%security
Furthermore, the resource management system should provide means to
ensure the integrity of the hardware resources taking part to the edge
infrastructure. Strategies such as enabling/disabling physical
interactions with the equipment should be considered.

% %
% Regarding administrators,  Administrators should be able to
% install and upgrade the aforementioned resource management
% services. After what, they can use those services to manage
% users, accesses, flavors (\ie available capacities of compute
% resources) and quotas.

% Regarding devops, they should be able to provision compute, network or
% storage resources like any traditional Cloud platform, supposing they
% are authorized to.

% In addition, admins and users share common expectations.
% % collect metrics
% First they want to monitor various metrics related to resources,
% users, and projects . This is used for instance for
% respectively managing quotas and listing existing resources.

\paragraph{L2: Operate/use several sites:}

%% TODO
% For instance, a user of \emph{Site 1} 
In L2, L1 features are considered but over multiple sites (at least
two).  This includes operations such as creating a new resource on a
particular site using resources from another one, managing several
resources or gathering information from various sites simultaneously.
%(interconnecting two VMs with a dedicated network).
Operations can be either intra-services (same service from different sites) or
inter-services (different services from different sites).

%
Concrete operations might consist for instance in configuring users'
access on a per-site basis, listing available VM images or pushing new
ones on multiple sites (intra-service operations)~\ldots From the
devops viewpoint, a user should be able to boot a VM on \emph{Site 1},
using an image defined in \emph{Site 2} (inter-service
operation). Similarly to L1, they both expect metrics collected from
several sites and collaborated mechanisms regarding the security (\eg
secret key sharing, network encryption).

Because collaboration between sites can be either explicit (\ie the
targeted sites are explicitly specified in the operation), or implicit
(\ie the resource management system is in charge of selecting
resources), we have defined two sub-levels as depicted in
\cref{tab:requirements} L2.1 and L2.2 respectively.  The implicit
manner suggests that policies (\eg performance objectives, energy
consumption) and constraints (\eg affinity, underlying hardware) are
provided by admins and users so that the resource management system takes
the right decisions regarding the defined desiderata and the state of
the infrastructure (\eg auto-scaling, relocating workloads between
sites, re-scheduling faulty resources across sites).
% should be elaborated and/or refs should be given here

\paragraph{L3: Robustness \wrt network split brains:}
The next level we define is related to the limited or intermittent
network connections between sites.  Similarly to the previous level,
L3 includes all L2 operations but with the possibility to face
\emph{network split brains} situations that (\ie, situations where the
infrastructure is partitioned due to communication failures).
%
\Cref{fig:fogedge-archi} depicts such a case where \emph{Site 1} for
instance is isolated from the other sites.  In this scenario, L1
operations should be feasible for administrators/devops that can reach
the site:~Although Site 1 is isolated from the rest of
the infrastructure, it should be possible for administrators/devops in
the same geographical area to perform L1 operations. Such a requirement make sense
as L1 operations are only related to one site and thus do not impact the other ones.
%
Regarding the other partition, \ie the one containing
several sites, L1 and L2 operations must be guaranteed for any set of
sites inside the partition.
%

We decided to refine L3 in two sub-classes in order to make a
distinction between the penalty a split brain can cause to resources
that have been already provisioned vs the management services that
enable to provision new resources or change existing ones. L3.1 is
related to features that allow existing workloads to continue to serve
local requests without being impacted (\eg. an apache server or a
storage backend should be able to satisfy requests coming from the
same geographical area even if their hosting servers cannot reach some
of the management services). L3.2 features are related to the
provisioning of new resources and other management operations as
described in the previous paragraph.
%
It is noteworthy to mention that delivering all L2
features will not be always possible at the L3.2 level because information
cannot be gathered in case of a disconnection.  For instance,
guaranteeing quotas across the infrastructure overall is impossible
without relaxing the consistency of the information. New approaches
will have to be proposed for such operations.

Finally, the intermittent network connectivity between edge resources
requires to have operations that enable sites to join and leave the
infrastructure.

% what about quotas? how to manage them in case of split-brain?
% suppose a user has a global quotas on several sites, it is impossible to
% determine its global consumption in case of split-brains?

\paragraph{L4: Multiple Cloud environments:}
Delivering a resource management system at large scale in a unified
manner is not conceivable from the software viewpoint.  Different
versions of each service will co-exist at the same time across the
whole infrastructure.  It is thus important to consider specific
requirements to deal with such a constraint.  L4 class gathers these
requirements.
%
More precisely, L4.1 considers L3 features when edge sites might be
operated by different versions of the same software stack.
%
L4.2 increases the complexity by adding into the big picture the
possibility to deal with different resource management systems,
including possibly different concepts (\eg OpenStack for VMs and
Kubernetes for Containers).

As a consequence, it is necessary to discover site's capabilities to
determine their relationship in order to only allow operations that are
possible.

\paragraph{L5: Multiple operators:}
L5 corresponds to the holy grail in terms of expected features.  In
addition to L4 features, it includes the possibility to use sites
owned by different operators. As depicted in \cref{tab:requirements},
we do not specify any requirement for administrators because an
operator would not share administrative rights to their sites with
other operators at the first sight. However, operators should be able
to collaborate to offer their sites' resources to any devops like it has
been done for a while for cellular networks. Requirements are more related to the collect and share
of relevant metrics enabling each operator to correctly establish bills.

%footer
\paragraph{Summary:}
As noticed in the beginning of this section, we choose to classify the
requirements through 5 levels, each one increasing the complexity in
terms of design and development constraints.  We underline that since
L4 and L5 both extend L3 respectively with multiple manager
environments, and multiple operators, they can be considered at the
same level in our classification and can be swapped as a consequence.

Although we tried to be exhaustive, this list of features could
probably be extended, in particular by considering different edge
infrastructure scenarios including smaller and limited devices.
However, we believe it is already valuable to deliver significant
insights on the design and implementation of a resource management
system for the edge computing. In the next section, we study whether a
system such as OpenStack can fulfill the L1, L2, and L3 levels.
The discussion of L4 and L5 is let as future work. 

%Considering now that the challenges are already numerous with L1, L2 and L3, we
%propose to drive the system design regarding these levels in the rest of the
%paper (leaving L4 and L5 for future works).

