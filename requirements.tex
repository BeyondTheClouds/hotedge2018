
\section{Administrators and Developers/End-users' Requirements}
\label{sec:requirements}

% \begin{figure}[t]
%   \centering
%   \def\svgwidth{\columnwidth}
%   \input{figures/sites.pdf_tex}
%   \caption{Partial representation of the targeted edge infrastructure, composed
%     of multiple sites. Each site is composed of many servers (represented by
%     black bullets) and have a set of users (depicted by black squares). The red
%     dashed line depicts a split-brain situation that separates the
%     infrastructure into $2$ partitions, isolating \emph{Site 1} from \emph{Site
%     2} and \emph{3}.}
%   \label{fig:fogedge-archi}
% \end{figure}

\begin{table*}
    \centering
        
\scriptsize
\begin{tabular}{@{} l L L L @{} >{\kern\tabcolsep}l @{}}
    \toprule
  %\begin{tabular}{@{} p{.25\textwidth} p{.25\textwidth}p{.25\textwidth}p{.25\textwidth} @{} >{\kern\tabcolsep}l @{}}    \toprule

    \emph{Levels} & \emph{Admin}& \emph{User} & \emph{Both} \\
    \midrule

    L1: Operate/use any site &
    Manage any site: &
    Provision on-demand on any site: &
    Collect metrics regarding a single site: \\ 

    (all operations are based in single site) &
    - install services &
    - compute resources&
    Security (secured communications, \\

    &
    - use services (users, flavors, quotas)&
    - network resources&
    auditing, integrity)\\

    &
    - upgrade services&
    - storage resources&
    Resiliency\\
    %\\% add space between levels

    \rowcolor{black!20}[0pt][0pt]
    L2: Operate/use several sites &
    &
    &
    L1 but over a set of sites\\

    \rowcolor{black!20}[0pt][0pt]
    (all operations cover at least two sites) &
    &
    &
    Collect metrics regarding many sites\\

    \rowcolor{black!20}[0pt][0pt]
    - L2.1 Explicit manner&
    Manage a specific set of sites &
    Provision on a specific set of sites &
    \\
    
    \rowcolor{black!20}[0pt][0pt]
    - L2.2 Implicit manner&
    Cross-site Autonomous management&
    Cross-site Autonomous provisioning&
    \\
    %\\% add space between levels

    L3: Robustness \wrt split brains &
    &
    &
    L1 for an isolated site\\ 

    (all operations cover one or many sites) &
    &
    &
    L1 and L2 for an isolated set of sites\\

    - L3.1 Application robustness&
    \hfill ------ \hfill &
    Access user's applications&
    Collect metrics regarding one or many sites\\

    - L3.2 Management service robustness&
    Manage one or many sites &
    Provision on one or many sites&
    \\
    %\\% add space between levels

    \rowcolor{black!20}[0pt][0pt]
    L4: Multiple Cloud environments &
    &
    &
    L3 with different types of IaaS managers\\ 

    \rowcolor{black!20}[0pt][0pt]
    (all operations cover one or many sites) &
    &
    &
    Discover sites' capabilities/compatibilities\\

    \rowcolor{black!20}[0pt][0pt]
    - L4.1 Different IaaS versions&
    Manage different IaaS versions&
    Provision on different IaaS versions&
    \\
    
    \rowcolor{black!20}[0pt][0pt]
    - L4.2 Different IaaS technologies&
    Manage different IaaS technos&
    Provision on different IaaS technos&
    \\
    %\\% add space between levels

    L5: Multiple operators &
    \hfill ------ \hfill &
    Provision on one or many sites&
    L4 with multiple operators\\ 

    (all operations cover one or many sites) &
    &
    &
    Collect metrics from different operators\\
    \bottomrule

\end{tabular}


    \caption{Classification of the requirements to administrate and use edge
    computing infrastructures in $5$ levels.}
    \label{tab:requirements}
\end{table*}

As previously mentioned, administrators and users of edge infrastructures
expect to get a set of high level mechanisms whose assembly results in a system
capable of operating and using a geo-distributed IaaS infrastructure.
Such mechanisms are provided various management services, each of which is in
charge of an aspect of the infrastructure management. In this paper, we
only consider a minimal set of a such services required to operate
infrastructure resources: Compute, Storage and Network managers, monitoring and
administrative tools~\cite{moreno2012csp}.
%However, considering each of these services independently is not
%sufficient enough to identify the challenges our community should deal with.
In this section, we classify the features administrators and
developers/end-users expect from those services, in the context of edge
computing, highlighting the differences with federated Cloud infrastructures.
\AL{Polish (in particular by adding a transition)}
Our classification is based on $5$ \emph{levels}, starting from the easiest
aspects, \ie interacting with a single site (considered in level 1 or L1), to
more complex aspects like managing multiple sites (L2), up to considering that
sites can be owned by different operators (L5).
\Cref{tab:requirements} summarizes the classification we detail in the following. 

\paragraph{L1: Operate/use any site}
As depicted in the second row of \cref{tab:requirements}, this level considers
the actions both administrators and users expect to perform on a single site.
Regarding the administration side, operating a single site like \emph{Site 1},
illustrated in \cref{fig:fogedge-archi}, requires that admins are able to
install and upgrade the resource management services mentioned previously.
After what, admins expect to use those services to manage users, accesses,
flavors (\ie available capacities of compute resources) or quotas, regarding
\emph{Site 1}.

Regarding end-users, they expect to be able to provision compute, network or
storage resources on any single site, supposing they are authorized to. For
instance, a user of \emph{Site 1} should be able to boot a VM in \emph{Site 2},
using images, flavors or networks defined in \emph{Site 2} -- as long as the
site is reachable.

In addition, admins and users share common expectations.
% collect metrics
First they expect metrics to be monitored (\eg metrics related to resources,
users, and projects) from any single site. This is used for instance for
respectively managing quotas and listing existing resources.
% security
Regarding the security aspects, communications are expected to be secured (\eg
isolated, encrypted) and actions must be logged for \emph{auditing}. Compared
to federated Clouds, Edge sites are potentially unmanned. As a consequence,
physical and application \emph{integrity} must be monitored at each site, and
autonomous corrective actions should be triggered when required.
% resiliency
Furthermore, compared to federated infrastructures, Edge infrastructures are
dynamic, \ie sites can join and leave the infrastructure, either on purpose, or
due to intermittent network access. As a consequence, tools must support
\emph{resiliency} regarding this churn effect.

\paragraph{L2: Operate/use several sites:}
We now consider in this level that the operations previously defined in L1
cover at least two sites. For instance, an administrator might desire to
configure users' access on a per-site basis. Regarding users, they expect
services from different sites to collaborate.
% is the following interesting regarding the scope of this paper?
% a reflection would need deeper analysis/explanations
Operations can be either intra-services (same service from different sites) or
inter-services (different services from different sites).
%
For instance, users might desire to boot a VM on \emph{Site 1}, using an image
defined in \emph{Site 2} (inter-service operation) or list VMs from multi-sites
(intra-service operation). Similarly to L1, they both expect metrics collected
from several sites and collaborated mechanisms regarding the security (\eg
secret key sharing).

We define here two sub-levels as depicted in \cref{tab:requirements}. Such
collaboration between sites can be either explicit (\ie the targeted sites are
explicitly specified), referenced as the sub-level L2.1, or implicit (L2.2).
The implicit manner suggests that policies (\eg performance objectives, energy
consumption) and constraints (\eg affinity, underlying hardware) are provided
by admins and users so that an edge-aware orchestrator takes the right
decisions regarding the users' desiderata and the state of the infrastructure
(\eg auto-scaling, relocating workloads between sites, re-scheduling faulty
resources across sites).
% should be elaborated and/or refs should be given here

\paragraph{L3: Robustness \wrt split brains:}
In this level, we now consider split brains which correspond to the situation
where the infrastructure is partitioned due to communication failures.
\Cref{fig:fogedge-archi} depicts such situation where \emph{Site 1} can be
isolated from the other sites. In such case, we distinguish two cases: (i) when
a partition is composed of an single site, like \emph{Site 1} in the figure,
all operations defined in L1 must be guaranteed regarding this site, supposing
it is reachable). For instance, a user in \emph{Site 1} must be
able to use \emph{Site 1}'s resources despite the split brain. Regarding
partitions containing several sites, L1 and L2 operations must be guaranteed
for any set of sites inside the partition.
% what about quotas? how to manage them in case of split-brain?
% suppose a user has a global quotas on several sites, it is impossible to
% determine its global consumption in case of split-brains?
Two sub-levels are proposed here: L3.1 considers the robustness of already
deployed resources (\eg a user should be able to access a deployed application
despite the split brain, if he can reach the related site), and is thus limited
to users, while the robustness of management services, treated in L3.2, is
required for both admins and users.
% tricky issue: management of split brain between collaborative services in
% different sites

\paragraph{L4: Multiple Cloud environments:}
Since edge infrastructures are dynamic infrastructures where sites can join and
leave, different versions of an IaaS manager can exist on different sites. One
requirement here is to consider L3 requirements between sites operated by
different manager versions (L4.1). Another requirement is to prevent vendor
lock-in between multiple generic IaaS manager technologies (\eg OpenStack,
CloudStack), and container orchestrators like Kubernetes (L4.2).
As a consequence, it is necessary to discover site's capabilities to determine
their compatibilities. For instance, it is not possible to migrate a VM from
\emph{Site 1} to \emph{Site 2} if the latter is managed by a container
orchestrator.

\paragraph{L5: Multiple operators:}
We now consider L4, plus the possibility that sites can be owned by different
operators. As depicted in \cref{tab:requirements}, no requirements is expected
by administrators since an operator would not share administrative rights to
their sites with other operators. However,
operators should be able to collaborate to offer their sites' resources to
users. Specific metrics should be collected at the scope of operators to
manage for instance billing between them.\\

%footer
Our classification of the requirements expected by both admins and users
highlights that the complexity to develop an edge infrastructure manager
significantly increases at each level. In fact, since L4 and L5 extends L3
respectively with multiple manager environments, and multiple operators, they
can be considered at the same level in our classification (and can be swapped
as a consequence).
Considering now that the challenges are already numerous with L1, L2 and L3, we
propose to drive the system design regarding these levels in the rest of the
paper (leaving L4 and L5 for future works).

