%%% Local Variables:
%%% mode: latex
%%% TeX-master: t
%%% End:
\section{OpenStack at The Edge}
\label{sec:system_design_considerations}
%\AL{We should add one sentence that justify why we have this section}
%
This section studies the use of OpenStack to control an Edge
infrastructure.  Such an analysis is meaningful as the OpenStack
codebase has been evolving to deal with large scale and multi-site
objectives for the two last years.  Precisely, the section evaluates
OpenStack with respect to its latest improvements regarding the requirements
defined in the previous section.
%in
%regards to the requirements defined in the previous section,
%highlighting why changes are mandatory despite recent improvements.


At coarse-grained, OpenStack has two types of nodes: data nodes
delivering XaaS capabilities (compute/storage/network, \ie data
plane); control nodes executing OpenStack services (\ie control
plane). Whenever users perform an action on OpenStack, the control
plane processes that action which ends on the data plane.  Among the
most important control plane services, one can cite \verb|keystone|,
\verb|nova|, \verb|glance|, and \verb|neutron|, whose respectively are
in charge of authentification, VM life cycle, their images and
associated networking.

Because OpenStack comes with several deployment alternatives and
because the edge sites considered in our study can host data and
control nodes, we discuss two scenarios. The first tries to mimic a
single DC environment by having a single OpenStack with all control
services on one site and data nodes on other sites.  The second
corresponds to a specific multi-region deployment, where one OpenStack
per Edge site is deployed.

\subsection{Centralized (Remote) Management}
\label{subsec:centralized_os}
In this scenario, OpenStack operates an Edge infrastructure
like a traditional single DC environment. %Cloud Computing one. 
The ``only'' difference is
related to the wide-area networking between the control nodes
and the compute nodes~\cite{www:openstack-wanwide}. The distinction between the different Edge
sites can be done by leveraging the concept of host aggregates
within OpenStack.

From the requirements' viewpoint, L1 and most L2 requirements can be
fulfilled in a straightforward way (because the infrastructure can be
spread over several network domains, some L2 operations including
specific network actions cannot be satisfied).  For L3, only L3.1
requirements can be satisfied, L3.2 expectations cannot be met due to
the considered deployment scenario. Most OpenStack services create and
manipulate logical objects that are persisted in shared databases
(DBs).  While this enables services to easily collaborate, it imposes
a permanent connectivity between compute nodes, services and their
DBs.  In other words, while this scenario provides a ``Single Pane of
Glass'' for administrator and DevOps, it has the drawback of being a
``Single Point of Failure'' preventing DevOps to use Edge resources in
case of network split brains.

\subsection{Multiple Regions}
In this scenario, each Edge site corresponds to a \emph{region}
in OpenStack terminology, which is a complete deployment of
OpenStack with all control services and a ``shared'' Keystone.
% (\ie the identify and service provider).
The main advantage of this deployment is related to the independency
of each site in case of network disconnections. In other words, in
case of a network split, each site can satisfy L1 features. The
downside relates to the fact that the current codebase does not
provide any mechanism to allow the collaboration between several
regions and thus L2 requirements cannot be met.~\endnote{Due to space
  limitation, we did not discuss a third scenario leveraging the
  OpenStack \emph{cells} concept. However, we underline that such an approach
  has the drawback of the two discussed scenarios.}


\subsection{Effective Collaboration is Needed}
Despite the fallibility of the network, and frequent isolation risks
of an Edge site from the rest of the infrastructure, the Edge
infrastructure may be able to meet L3 requirements. This
is achieved by supposing a collaboration a la \emph{peer-to-peer},
that is, an Edge site always serves local resources and collaborates
with other Edge sites if need be. To develop such a resource
management system, the developer has two fundamental design options:
\emph{top-down} or \emph{bottom-up}. Both designs impact how to handle
the collaboration needed by the system.

A top-down collaboration design implements the collaboration by
federating VIMs' API. It leverages on existing VIMs as they are made
available today without introducing modifications/extensions. Examples
of approaches following a \emph{top-down} design are:
ONAP~\cite{onap}, Kingbird~\cite{kingbird},
FogBow~\cite{brasileiro2016fogbow} and
p2p-OpenStack~\cite{ericsson-p2p}

A bottom-up collaboration design lays on making VIMs
mechanisms/services directly collaborative~\cite{7923796}. For example, having two
OpenStack Nova services able to cooperate and communicate directly
would be a realization of a bottom-up design. Such design implies
either the modification/extension of existing VIMs or the creation of
a completely new system.