
%%% Local Variables:
%%% mode: latex
%%% TeX-master: t
%%% End:

\section{Design Considerations}
\label{design_considerations}
This paper considers an Edge Infrastructure that is composed of
hundreds of micro Data Centers named Edge Site. Each Edge Site
provides control and compute capabilities (see~\ref{sec:requirements})
on top of dozens of servers. They \emph{collaborate} with each-other
to shape up the Edge Infrastructure and a Wide Area Network (WAN)
interconnects them through both wired and wireless network links. This
results in up to 300ms of RTT between two Edge Sites and common
disconnections.

Despite the fallibility of the network, and frequent isolation risks
of an Edge Site form the rest of the Infrastructure (\ie network
split-brain), the Edge Infrastructure may be kept sustainable. This is
achieved by supposing a collaboration a la peer-to-peer, that is, an
Edge Site always serves local resources and collaborates with other
Edge Sites if needed. Thus, a user can make a SSH on his VM during a
network split-brain if he gets a local access to the Edge Site. The
same user can also start a VM using the VMI of an other Edge Site if
the collaboration can be established.

When it comes to develop this resource management system for the Edge
computing, the developer gets two possibilities: a \emph{top-down} or
\emph{bottom-up} design. Both designs impact how to handle the
collaboration needed by such a system.

\paragraph{Top-Down Collaboration}
The top-down design implements the collaboration required by by
federating IaaS managers' API. todo

\paragraph{Bottom-Up Collaboration} todo
