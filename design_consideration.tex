
%%% Local Variables:
%%% mode: latex
%%% TeX-master: t
%%% End:

%\section{Design Considerations}
%\label{design_considerations}

\section{System Design}
\label{system_design_considerations}
This paper considers an Edge Infrastructure that is composed of
hundreds of micro Data Centers named Edge Site. The first approach to operate such a massively distributed infrastructure is to take 


\subsection{Considerations}
\label{design_considerations}
 Each Edge Site
provides control and compute capabilities (see~\ref{sec:requirements})
on top of dozens of servers. They \emph{collaborate} with each-other
to shape up the Edge Infrastructure and a Wide Area Network (WAN)
interconnects them through both wired and wireless network links. This
results in up to 300ms of RTT between two Edge Sites and common
disconnections.

Despite the fallibility of the network, and frequent isolation risks
of an Edge Site from the rest of the Infrastructure (\ie network
split-brain), the Edge Infrastructure may be kept sustainable. This is
achieved by supposing a collaboration a la peer-to-peer, that is, an
Edge Site always serves local resources and collaborates with other
Edge Sites if needed. Thus, a user can make a SSH on his VM during a
network split-brain if he gets a local access to the Edge Site. The
same user can also start a VM using the VMI of an other Edge Site if
the collaboration can be established.

When it comes to develop this resource management system for the Edge
computing, the developer has two fundamental design options: a \emph{top-down} or
\emph{bottom-up}. Both designs impact how to handle the
collaboration needed by such a system.

\paragraph{Top-Down Collaboration}
Design option that implements the collaboration required by
federating IaaS managers' API. In other words, it leverages on existing IaaS platforms as they are made available today without introducing modifications/extensions. Examples of approaches following a \emph{top-down} design are: ONAP~\cite{onap}, Kingbird~\cite{kingbird}, FogBow~\cite{brasileiro2016fogbow} and p2p-OpenStack~\cite{ericsson-p2p}


\paragraph{Bottom-Up Collaboration} Design option that lays on making IaaS mechanisms/services directly collaborative. For example, having two OpenStack Nova services able to cooperate and communicate directly would be a realization of a \emph{bottom-up} design. Such design option implies either the modification/extension of existing IaaS platforms or the creation of a completely new system. Examples of approaches following a \emph{bottom-up} design are:
\AL[Comment by Joao]{Any reference to existing approaches? Guys, do you have, for example, references to your keystone shared db work? Could cells be considered as an example? }

On the one hand, \emph{top-down} options have been the most explored, but can these alone fulfil all expected requirements? On the other hand, \emph{bottom-up} options seem to disruptive the design principles of most existing IaaS platform, but is that in fact the case?
