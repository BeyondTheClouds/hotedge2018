\section{Introduction}
\label{sec:intro}


% % Edge Infrastructures are the next platform
% With the emergence of Network Function Virtualization (NFV)
% technologies as well as Internet of Things (IoT) and augmented/virtual
% reality (AR/VR) applications, Cloud and Network communities are
% advocating for going towards massively distributed small sized
% infrastructures that are deployed at the edge of the network.
% %
% Referred to as the Edge Computing paradigm, this model aims at
% delivering Cloud Computing capabilities closer to end-users, their
% related devices, and applications.
% %

While several academic studies have been highlighting the advantages
of Edge Computing paradigm in various
domains~\cite{bonomi2012fog,zhang2015cloud,yi2015fog,shi2016edge,satyanarayanan2017emergence},
%%
progress on how operating and using infrastructures that serve it are
only marginal.  As an example, current solutions such as Amazon
Lambda@Edge~\cite{amazon:lambda-edge} or Akamai
Cloudlet~\cite{akamai:cloudlets} to name a few are rather close to the
initial fog proposal that allowed to run specific domain applications
on infrastructures composed of NFV-enabled hardware (at the edge) and
centralized clouds~\cite{bonomi2012fog}.  In other words, current Edge
Infrastructure-as-a-Service solutions do not allow to run stateful
workloads in isolated environments such as containers or virtual
machines (VMs).
%
This is surprising as devops expect to find most features that make the sucess of
current Cloud Computing solutions.

% Our community should tackle the challenge of delivering a general
% resource management system to operate and use Edge Computing
% infrastructure. Something that will first, let an operator aggregates,
% supervises and exposes the massively distributed resources of the
% infrastructure and second, let a developer implements new kinds of
% services on top of that infrastructure that may be deployed and
% managed on demand.

%% Thus, the development of a general resource management system to
%% operate and use Edge Computing infrastructures is still an open
%% question.

%% %%  We need a resource management system that enables administrators to
%% %% operate and end-users to use edge resources.
%% %% a resource management system that will enable, on the one hand,
%% %% an operator to aggregate, supervise and expose such massively
%% %% distributed resources and, on the other hand, a developer to implement
%% %% new kinds of services that may be deployed and managed on demand is
%% %% still an open question.

%% %%  Delivierng en Edge IaaS system is complex
%% Domain specific solutions~\cite{bonomi2012fog} allow IoT applications
%% to run on infrastructures composed of NFV-enabled hardware (at the
%% edge) and existing centralized clouds. However, these solutions do not
%% allow to run workloads in isolated environments such as containers or
%% virtual machines (VMs).
%% %
%% This is rather critical as developers/end-users expect to find
%% most features that makes the success of current Cloud Computing solutions.

%% Requirements:
%%
%% 1. il y a des fonctionnalités spécifiques au edge
%%
%% 2. on a étudié différentes stratégies de design d'une IaaS manager
%%    qui implementerait nos requieremtn avec comme *contraintes de
%%    réutilier au plus les VIM*.

%% Les implem essaye de réutiliser les VIM pour minimiser l'effort de
%% dvlp. Elle font ceci par aggrega des API, mais le problème est
%% qu'elles se limite à être des gestionnaires de resources (cf k8s
%% discussion avec Adrien) alors que un VRAIE IaaS manager doit
%% considérer plus de chose (donner un exemple de req).

%On this basis,
%
The ETSI Mobile Edge Computing Industry Specification Group
defined in 2016 a software architecture to orchestrate distinct
independent Cloud systems or, in their terminology, Virtual
Infrastructure Managers (VIM)~\cite{7574435}.
%
The idea consists in federating VIMs of the several Data Centers that
compose the Edge infrastructure.  By reusing VIMs, the ETSI group targets an Edge
Computing resource management that behaves in a same fashion as Cloud
Computing, while mitigating development requirements.
%
Although, there is no implementation available, the idea of federating
VIMs seems promising since several projects have been built on similar
concepts. The ONAP framework~\cite{onap}, an industry-driven solution,
enables the orchestration and automation of virtual network functions
across distinct VIMs. From the academic side, the FogBow
solution~\cite{brasileiro2016fogbow} aims to support large federations
of Infrastructure-as-a-service (IaaS) providers. More recently, NIST
initiated a collaborative effort with the IEEE to advance Federated
Cloud through the development of a conceptual architecture and a
vocabulary\endnote{\url{https://collaborate.nist.gov/twiki-cloud-computing/bin/view/CloudComputing/FederatedCloudPWGFC}
  (Valid on March 2018).}.

While all these projects provided valuable contributions, they have
been all designed by considering the devops' perspective: they provide
abstractions to manage the life cycle of geo-distributed applications
but do not address administrators' requirements.
%
Edge Computing infrastructures differ from federated cloud systems
under various aspects~\cite{openstack:whitepaper}.  For instance, Edge
sites are potentially unmanned, and therefore must be administered
remotely. The resource management system should also be designed to
cope with intermittent network access to the site~\ldots
%
Ultimately, distinct operators might be interested in interconnecting their infrastructures (like network peering).
%

To favor the advent of Edge Computing, our community should take part
to current discussions and actions in order to deliver a well-suited
resource management system:~A system that will first, let an operator
aggregates, supervises and exposes the massively distributed resources
of the infrastructure and second, let devops implement new kinds
of services on top of that infrastructure that may be deployed and
managed on demand.

% Designing and implementing a complete software stack to enable
% administrators to operate, and developers to use, distinct edge sites
% like a global Cloud Computing infrastructure is a difficult challenge
% for our community. However it is a challenge we should tackle soon to
% favor the advent of the Edge Computing paradigm.

In this paper, we present reflections to initiate discussions through our community.

\begin{itemize}
\item First, we introduce a classification of expected features from
  both administrators and devops. This classification is valuable to
  identify missing mechanisms in opensource resource management systems.
 % Moreover, it should deliver significant insights on the
 % design and implementation of a resource management system for the
 % edge computing.
\item Second, on the basis of the identified requirements, we discuss
  insights regarding how an edge computing resource management system
  should be designed. In particular, we study \emph{pros} and
  \emph{cons} of \emph{top-down} and \emph{bottom-up} approaches. The
  former consists in interacting with each site only through the
  exposed APIs such as the aforementioned federated approaches. The
  latter aims at revising internal mechanisms of VIMs to enable native
  collaborations.
  \end{itemize}

\begin{figure}[t]
  \centering
  \includegraphics[width=\columnwidth]{./figures/figure_fog.pdf}
  \caption{Edge Computing Infrastructure~\cite{7923796}.}
    {\small It can be extended with Cloud Computing
    platforms when relevant.  The red dashed lines depicts a split-brain situation that isolates
    \emph{Site 1} from other sites.}
  \label{fig:fogedge-archi}
\end{figure}

Because there are as many edge infrastructures as use-cases, we
highlight that the edge infrastructure we considered to conduct this
study is composed from the hardware viewpoint of several
geo-distributed micro DCs (up to hundreds)
composed of up to one hundred servers (approximately two racks in terms of space).
Figure~\ref{fig:fogedge-archi} gives an overview of this
infrastructure. The expected latency as well as the bandwidth between
each element may fluctuate significantly, in particular because the
network links can be either wired or wireless (represented by plain
and dashed lines on the figure). Moreover, short disconnections
between edge sites may occur leading to temporary network split brain
situations~\cite{4456903}. Finally, it might be possible
to consider additional micro DCs deployed at the Extreme Edge, within
private institutions or even public transports.
%
From the software point of view, in order to simplify the model and
without loss of generality, we mainly assume the OpenStack software
suite~\cite{openstack:www} as the default VIM.  After seven years of intensive developments, OpenStack has
become the de facto open-source solution to operate, supervise and use
IaaS infrastructures.
%More recently it has increased its effort to address edge computing cases.
%The OpenStack community gathers more than 500
%organizations, including large groups, in particular key actors of
%edge infrastructures such as ATT, Verizon~\ldots.

The remainder of the paper is organized as
follows. Section~\ref{sec:requirements} presents major features
expected by administrators and devops, highlighting in particular
the main differences w.r.t federated infrastructures.
Section~\ref{sec:system_design_considerations} deals with two
deployment scenarios based on the today's system capabilities,
highlighting the need for an effective system collaboration across the
entire edge infrastructure. Section~\ref{sec:design_discussion}
introduces two approaches that open source communities could follow to
design and develop a resource management system dedicated to edge
environments. It discusses in particular \emph{pros}/\emph{cons} of
both approaches and points out a way forward. Finally,
Section~\ref{sec:conclusion} concludes this paper.


% In this paper, we provide a list of expected features for both administrators and admins.
% We investigate whether a stack such as OpenStack, the defacto open-source standard can fullfil these requirements
% We finally discusss two possibles ways for moving forward: top/down vs bottom/up.
