
\section{Introduction}
\label{sec:intro}


% Edge Infrastructures are the next platform
With the emergence of Internet of Things applications (IoT) and new
usages related Network Function
Virtualization technologies (NFV), Cloud and Network communities are now advocating
for going towards massively distributed small sized infrastructures
that are deployed at the edge of the network.
%
Referred to as the Edge Computing paradigm, this model has been attracting
growing interest for the last few years as it enables delivering Cloud
Computing capabilities closer to end-users, their related devices, and
applications.
%

%% A lot of academics works focus on how to leverage these
%% infrastructures to address specicifc use-cases
While several academic studies have highlighed the advantages of Edge
Computing infrastructures in different
domains~\cite{bonomi2012fog,zhang2015cloud,7574435,satyanarayanan2017emergence},
the development of
%%  We need a resource management system that enables administrators to
%% operate and end-users to use edge resources.
an appropriate management system that will enable, on the first hand,
an operator to aggregate, supervise and expose such massively
distributed resources and, on the other hand, developers to implement
new kinds of services that may be deployed and managed on demand is
still an open question.

%%  Delivierng en Edge IaaS system is complex
To the best of our knowledge, only a few activities have been
investigating how such a system can be designed overall.  With their
IOx and Fog Director software solutions~\cite{bonomi2012fog}, Cisco allows IoT
applications to run on infrastructures composed of NFV-enabled
hardware (at the edge) and existing centralized clouds. However, their
solution does not allow to run Virtual Machines  and does not compete
with existing IaaS platforms.
%
Although Edge infrastructures significantly differ from traditional Cloud ones
regarding heterogeneity, dynamicity and the potential massive
distribution of resources and networking environment, administrators and end-users expect the same capabilities.
To mitigate the complexity of a complete stack, most advanced proposals have been focusing
on orchestrating applications accross distinct independent cloud
systems (\aka virtual infrastructure managers).
%
Among the major projects, one can cite the architecture
defined by the European Telecommunications Standards Institute (ETSI)
MEC Industry Specification Group~\cite{7574435} or more recently the
ONAP framework~\cite{onap}, which is supported by several industrial groups. 
From the academic side, brokering approaches have been also proposed.
For instance, the FogBow proposal~\cite{brasileiro2016fogbow} aims to support large
federations of Infrastructure-as-a-service (IaaS) cloud providers.

While all these projects have extended current interfaces (APIs) to
turn location into a first-class citizen, and allow developpers to
manage edge application life cycle in geo-distributed manner, they do not
provide a well-suited software stack to operate and use edge infrastructures like cloud computing ones. 

This paper proposes several
contributions to make progress on the aformentioned question.
\begin{itemize}
\item A list of expected features from bothe the administrators and end-users' perspectives;
\item A discussion of pros/cons of top/down 
\end{itemize}

The remaining of the paper ...



% In this paper, we provide a list of expected features for both administrators and admins.
% We investigate whether a stack such as OpenStack, the defacto open-source standard can fullfil these requirements
% We finally discusss two possibles ways for moving forward: top/down vs bottom/up.


% TODO: talk about:
% The NIST Cloud Computing Program (NCCP) recognized, in the USG Cloud Computing Standards and Technology Roadmap (NIST SP 500-293), the importance of advancement and development of frameworks to support seamless implementation of federated community cloud environments. In August 2017, the NCCP Public Working Group (PWG) on Federated Cloud (PWGFC) initiated a collaborative effort with the IEEE P2302 group, Standards for Intercloud Interoperability and Federation (SIIF), to advance Federated Cloud through the development of a conceptual architecture and a vocabulary. The IEEE P2302 will develop a standard based on the NIST model.  Additional information can be found at https://collaborate.nist.gov/twiki-cloud-computing/bin/view/CloudComputing/FederatedCloudPWGFC  

