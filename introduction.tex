
\section{Introduction}
\label{sec:intro}


% Edge Infrastructures are the next platform
With the emergence of Internet of Things applications (IoT) and new
usages related Network Function
Virtualization technologies (NFV), Cloud and Network communities are now advocating
for going towards massively distributed small sized infrastructures
that are deployed at the edge of the network.

Referred to as the Edge paradigm, this model has been attracting
growing interest for the last few years as it enables to deliver cloud
computing capabilities closer to end-users, their related devices, and
applications.

% A lot of academics works focus on how to leverage these infrastructures to address specicifc use-cases
While several academic studies have highlighed the advantages of Edge
Computing infrastructures in different domains, the development of 
% We need a resource management system that enables administrators to operate and end-users to use edge resources.
an appropriate management system that will enable, on the first hand, an
operator to aggregate, supervise and expose such massively distributed
resources and, on the other hand, developers to implement new kinds of services
that may be deployed and managed on demand is still an open question.

Designing a well-suited management system
is a challenging task because Edge infrastructures significantly
differ from traditional Cloud ones regarding heterogeneity,
dynamicity and the potential massive distribution of resources and
networking environments.


% In this paper, we provide a list of expected features for both administrators and admins.
% We investigate whether a stack such as OpenStack, the defacto open-source standard can fullfil these requirements
% We finally discusss two possibles ways for moving forward: top/down vs bottom/up.


% TODO: talk about:
% The NIST Cloud Computing Program (NCCP) recognized, in the USG Cloud Computing Standards and Technology Roadmap (NIST SP 500-293), the importance of advancement and development of frameworks to support seamless implementation of federated community cloud environments. In August 2017, the NCCP Public Working Group (PWG) on Federated Cloud (PWGFC) initiated a collaborative effort with the IEEE P2302 group, Standards for Intercloud Interoperability and Federation (SIIF), to advance Federated Cloud through the development of a conceptual architecture and a vocabulary. The IEEE P2302 will develop a standard based on the NIST model.  Additional information can be found at https://collaborate.nist.gov/twiki-cloud-computing/bin/view/CloudComputing/FederatedCloudPWGFC  

