
\section{Introduction}
\label{sec:intro}


% Edge Infrastructures are the next platform
With the emergence of Internet of Things applications (IoT) and new
usages related Network Function
Virtualization technologies (NFV), Cloud and Network communities are now advocating
for going towards massively distributed small sized infrastructures
that are deployed at the edge of the network.
%
Referred to as the Edge Computing paradigm, this model has been attracting
growing interest for the last few years as it enables delivering Cloud
Computing capabilities closer to end-users, their related devices, and the applications.
%

%% A lot of academics works focus on how to leverage these
%% infrastructures to address specicifc use-cases
While several academic studies have highlighed the advantages of Edge
Computing infrastructures in various
domains~\cite{bonomi2012fog,zhang2015cloud,yi2015fog,shi2016edge,satyanarayanan2017emergence},
the development of
%%  We need a resource management system that enables administrators to
%% operate and end-users to use edge resources.
a resource management system that will enable, on the first hand,
an operator to aggregate, supervise and expose such massively
distributed resources and, on the other hand, developers to implement
new kinds of services that may be deployed and managed on demand is
still an open question.

%%  Delivierng en Edge IaaS system is complex
With their IOx and Fog Director software
solutions~\cite{bonomi2012fog}, Cisco allows IoT applications to run
on infrastructures composed of NFV-enabled hardware (at the edge) and
existing centralized clouds. However, their solution does not allow to
run workloads in isolated environments such as containers or virtual
machines (VMs). 
%
This is rather critical as administrators and end-users expect to find
most features that makes the success of the Cloud Computing paradigm
and even if Edge infrastructures significantly differ regarding
heterogeneity, resiliency and the potential massive distribution of
resources and networking environment.

In 2016, the European Telecommunications Standards Institute (ETSI) MEC
Industry Specification Group defined  a software architecture
to orchestrate distinct independent cloud systems, \aka virtual
infrastructure managers in their terminology
(VIM)~\cite{7574435}. While the proposal should allow the deployment
of VMs at the edge, there is no implementation available yet.  At the
same time, a consortium composed of several industrial groups proposed
the ONAP framework~\cite{onap} that enables the orchestration and
automation of virtual network functions across distinct VIMs.  From
the academic side, federated approaches have been also investigated.
For instance, the FogBow proposal~\cite{brasileiro2016fogbow} aims to
support large federations of Infrastructure-as-a-service (IaaS) cloud
providers. More recently, the NIST Cloud Computing Program (NCCP)
%Public Working Group has recognized the importance of advancement and development of
%frameworks to support seamless implementation of federated community
%cloud environments. In August 2017, the NCCP
%Public Working Group
% (PWG) on Federated Cloud (PWGFC)
initiated a collaborative effort with the IEEE P2302 group (Standards for Intercloud Interoperability and
Federation), to advance Federated Cloud through the development
of a conceptual architecture and a vocabulary.
%The IEEE P2302 will
%develop a standard based on the NIST model.  Additional information
%can be found at
\footnote{\url{https://collaborate.nist.gov/twiki-cloud-computing/bin/view/CloudComputing/FederatedCloudPWGFC} (Valid on March 2018).}
%
While all these projects propose extensions to %current interfaces
% (APIs) to turn location into a first-class citizen, and
allow end-users/developers to manage edge applications life cycle,
they do not provide a well-suited software stack to enable
administrators to operate, and developers to use, distinct edge sites
like a global Cloud Computing infrastructure. In other words, there is
currently not well-suited solutions to easily deliver cloud computing
services at the edge.
%
%

In this paper, we present preliminary reflexions regarding how such a
system can be delivered. To this aim, we propose~:
\begin{itemize}
\item A classification of the features that are expected from both
  administrators and end-users of edge computing infrastructures
  ; %, which higlights in particular why edge infrastructures differ
    %from federated ones ;
\item An analyzis of pros and cons of possible design strategies
  (top/down and bottom up).
\end{itemize}

The edge infrastructure we considered to conduct this study is
composed of up to hundreds of micro DCs, which are themselves composed
of up to one hundred servers (up to two racks).  Figure 1 gives an
overview of this infrastructure with several micro/nano DCs. The
expected latency as well as the bandwidth between each element may
fluctuate significantly, in particular because the network links can
be either wired or wireless (represented by plain and dashed lines on
the figure). Finally, it might be possible to consider additional DCs
deployed at the Extreme Edge, within a private institution.


The remaining of the paper is organized as
follows. Section~\ref{sec:requirements} presents major features expected by administrators and developers, higlightling in particular the main differences
w.r.t federated infrastructures.
Section~\ref{sec:design_considerations} introduces two approaches
that opensource communities could follow to design and develop a IaaS
toolkit, \aka a resource management system, dedicated to edge
environments. Section~\ref{sec:design_discussion} discusses pros/cons
of both approaches. Existing solutions that might serve as
starting points, are discussed in
Section~\ref{sec:related_work}. Finally, Section~\ref{sec:conclusion} concludes this paper.


% In this paper, we provide a list of expected features for both administrators and admins.
% We investigate whether a stack such as OpenStack, the defacto open-source standard can fullfil these requirements
% We finally discusss two possibles ways for moving forward: top/down vs bottom/up.

