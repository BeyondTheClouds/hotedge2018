
\subsection{Non-OpenStack approaches}
\label{subsec:non-os}

\subsubsection{ONAP}

The Open Network Automation Platform (ONAP) \cite{web:onap} is a framework to
design and instantiate Virtual Network Functions (VNFs) on top of Virtual
Infrastructure Managers (VIMs), like OpenStack, Microsoft Azure or Amazon Web
Services. Driven by the Linux Foundation since 2017, this active project is
built on a modular architecture where each module is in charge of the designing
or the orchestration of VNFs. For instance, some modules are in charge of
provisioning compute and network resources, while others are in charge of
scheduling, authentication or storing images.

%
%For instance, the Master Service Orchestration (MSO) module coordinates the
%actions to deploy VNFs, while the Data Collector, Analytics and Events (DCAE)
%module is in charge of collecting metrics and raises alarms to other modules.
%

More particularly, ONAP relies on the multi-VIM module to orchestrate multiple
independent infrastructure managers. This module provides an abstraction of the
underlying VIMs and acts as a request broker between ONAP's modules and VIMs. To
that end, it is itself composed of sub-modules to manage inter and intra-VIM
actions (e.g. inter-VIM scheduling, network overlays).

